% \PassOptionsToClass{negatif}{m53beamer}
% \PassOptionsToClass{simple}{m53beamer}
% \PassOptionsToClass{handout}{m53beamer}
\documentclass[11pt]{m53beamer}

% ---------------
\title{M53 - Partie 3}
%\author{Kroum Tzanev}
\date{novembre 2015}
% ---------------

\begin{document}

% ------- Le titre --------
\begin{frame}
  \titlepage
\end{frame}

% =======================================
\section{Cônes et cylindres}
% =======================================

% ~~~~~~~~~~~~~~~~~~~~~~~~~~~~~~~~~~~~~~~
\subsection{Cône}
% ~~~~~~~~~~~~~~~~~~~~~~~~~~~~~~~~~~~~~~~

% -----------
\begin{frame}{Définition d'un cône}
  Soient $\ens{E}$ un espace affine sur $\mathbb{R}$ et $O \in \ens{E}$.\pause
  \begin{definition}
    Un ensemble $\ens{C}$ est dit \alert{cône de centre $O$}\,$\in \ens{C}$ si pour tout $M \in \ens{C}\setminus\!\{O\}$, $\ens{C}$ contient la droite $\affspan{O,M}$.\pause{}
    Et on dit que $\ens{C}$ est de \alert{base} $\ens{B} \subset \ens{C}$ si pour tout $M \in \ens{C}\setminus\!\{O\}$, $\card\big(\affspan{O,M} \cap \ens{B}\big)=1$.
  \end{definition}\pause
  \begin{defprop}
    Soient $O \in \ens{E}$ et $\ens{F} \subset \ens{E}$ non vide, on dit que $\ens{C}$ est \alert{le cône de centre $O$ engendré par $\ens{F}$} si l'une des conditions équivalentes suivantes est satisfaite:
    \begin{enumerate}[<+(1)->]
         \item $\ens{C}$ est le plus petit cône de centre $O$ contenant $\ens{F}$.
         \item $\ens{C}=\bigcup_{M \in \ens{F}} \affspan{O,M}$.
         \item $\ens{C}=\bigcup_{\lambda \in \mathbb{R}} H_{O,\lambda}(\ens{F})$, où $H_{O,\lambda}$ est l'homothétie de centre $O$ et de rapport $\lambda$.
       \end{enumerate}
  \end{defprop}
\end{frame}
% -----------
\begin{frame}{Le cône standard}
  On se place dans $\mathbb{R}^{3}$, vu comme espace affine.
  \begin{defprop}
    \alert{Le cône standard} $\ens{C}$ de $\mathbb{R}^{3}$ est défini par l'une des conditions équivalentes:
    \begin{enumerate}[<+(1)->]
         \item $\ens{C}=\{x^{2}+y^{2}=z^{2}\}$\pause$=\{x^{2}+y^{2}-z^{2}=0\}$.
         \item $\ens{C}$ est le cône de centre $0$ engendré par le cercle $\{x^{2}+y^{2}=1, z=1\}$.
         \item $\ens{C}$ est l'ensemble obtenu par la rotation autour de l'axe $Oz$ de la droite vectorielle $\{x=z, y=0\}$.
       \end{enumerate}
  \end{defprop}\pause
  Les intersections du cône standard avec les hyperplans non vectoriels sont les \textbf{coniques} \uncover<+(1)->{(\textbf{ellipses}, \textbf{paraboles} et \textbf{hyperboles}\uncover<+(1)->{, éventuellement dégénérées en des droites ou en un point})} qui vont être l'objet d'étude de cette troisième partie du cours.
\end{frame}
% -----------
\begin{frame}{Exemples moins standards}
  \begin{enumerate}[<+(1)->]
    \item L'espace tout entier $\mathbb{R}^{n}$ est un cône de centre $0$ et de base «~la~moitié~» de la sphère $\mathbb{S}^{n-1}$.
    \item Toute réunion d'espaces vectoriels est un cône de centre $0$.
    \item $\ens{C}=\{x^{2}=yz\}$ est un cône de centre $0$ et de base l'ellipse $\{x^{2}+(y-1)^{2}=1, y+z=2\}$.
    \item Les images directes et inverses d'un cône par des applications affines sont des cônes.
  \end{enumerate}
\end{frame}

% ~~~~~~~~~~~~~~~~~~~~~~~~~~~~~~~~~~~~~~~
\subsection{Cylindre}
% ~~~~~~~~~~~~~~~~~~~~~~~~~~~~~~~~~~~~~~~

% -----------
\begin{frame}{Définition d'un cylindre}
  Soient $\ens{E}$ un espace affine et $\ev{F} \subset \ev{E}$ un sous-espace vectoriel de directions.\pause
  \begin{definition}
    Un ensemble non vide $\ens{C}$ est dit \alert{cylindre de direction $\ev{F}$} si pour tout $M \in \ens{C}$, $\ens{C}$ contient le sous-espace affine $M+\ev{F}$.\pause{} Et on dit que $\ens{C}$ est de \alert{base} $\ens{B} \subset \ens{C}$ si pour tout $M \in \ens{C}$, $\card\big( (M+\ev{F}) \cap \ens{B}\big)=1$.\pause\newline
    Le plus souvent la direction $\ev{F}$ est une droite.
  \end{definition}\pause
  \begin{defprop}
    Soient $\ev{F} \subset \ev{E}$ un sous-espace vectoriel de directions, et $\ens{B} \subset \ens{E}$ non vide, on dit que $\ens{C}$ est \alert{le cylindre de direction $\ev{F}$ engendré par $\ens{B}$} si l'une des conditions équivalentes suivantes est satisfaite:
    \begin{enumerate}[<+(1)->]
         \item $\ens{C}$ est le plus petit cylindre de direction $\ev{F}$ contenant $\ens{B}$.
         \item $\ens{C}=\bigcup_{M \in \ens{B}} (M+\ev{F})$.
         \item $\ens{C}=\bigcup_{\vv{v} \in \ev{F}} T_{\vv{v}}(\ens{B})$, où $T_{\vv{v}}$ est la translation par le vecteur $\vv{v}$.
       \end{enumerate}
  \end{defprop}
\end{frame}
% -----------
\begin{frame}{Le cylindre standard}
  On se place dans $\mathbb{R}^{3}$, vu comme espace affine.
  \begin{defprop}
    \alert{Le cylindre standard} $\ens{C}$ de $\mathbb{R}^{3}$ est défini par l'une des conditions équivalentes:
    \begin{enumerate}[<+(1)->]
      \item $\ens{C}=\{x^{2}+y^{2}=1\}$.
      \item $\ens{C}$ est le cylindre de direction $Oz$ engendré par le cercle $\{x^{2}+y^{2}=1, z=0\}$.
      \item $\ens{C}$ est l'ensemble obtenu par la rotation autour de l'axe $Oz$ de la droite vectorielle $\{x=1, y=0\}$.
    \end{enumerate}
  \end{defprop}\pause
  L'intersection du cylindre standard avec un hyperplan est une ellipse\pause, deux droites\pause, une droite\pause, ou bien vide.
\end{frame}

% ~~~~~~~~~~~~~~~~~~~~~~~~~~~~~~~~~~~~~~~
\subsection{Courbes de niveaux}
% ~~~~~~~~~~~~~~~~~~~~~~~~~~~~~~~~~~~~~~~

% -----------
\begin{frame}{Définition de courbe de niveau}
  Soit une application $f:A \to B$ entre ensembles.
  \begin{definition}
    Soit $k \in B$, l'ensemble $\ens{L}_{k}=f\,^{-1}(\{k\})$ est appelé \alert{la courbe de niveau $k$} de $f$.\pause{} On peut également l'appeler \alert{la ligne de niveau $k$} ou \alert{l'ensemble de niveau $k$} de $f$, et des notations alternatives sont $\ens{L}_{k}(f\,)$ ou $\ens{L}_{f,k}$.
  \end{definition}\pause
  La terminologie «courbe de niveau» prend tout son sens quand $f:\mathbb{R}^{2} \to \mathbb{R}$ est différentiable et $k$ une valeur régulière.
\end{frame}
% -----------
\begin{frame}{Exemples de courbes de niveaux}
  \begin{enumerate}[<+(1)->]
    \item Pour $f(x,y)=x^{2}+y^{2}$, $\ens{L}_{k}$ est un cercle, un point, ou vide, selon que $k$ soit positif (valeur régulière), nul ou négatif.
    \item Pour $f(x,y)=ax+by$, avec $(a,b)\neq(0,0)$, les $\ens{L}_{k}$ sont les droites affines de direction la droite vectorielle $\ens{L}_{0}$.
    \item Plus généralement tout ensemble défini par des équations $\{ x \in \ens{E} \,|\, f_{1}(x)=b_{1},\ldots,f_{p}(x)=b_{p}\}$ est la courbe de niveau $(b_{1},\ldots,b_{p})$ de l'application $f(x)=\big(f_{1}(x),\ldots,f_{p}(x)\big)$.
  \end{enumerate}
\end{frame}
% -----------
\begin{frame}{Image d'une courbe de niveau}
  \begin{proposition}
    Soit un isomorphisme $\phi:A\isoto B$ et $f:A\to C$, alors l'image de $\ens{L}_{k}(f\,)$ par $\phi$ est $\ens{L}_{k}(f\circ \phi^{-1}\,)$.
  \end{proposition}\pause
  Exemples :
  \begin{enumerate}[<+(1)->]
  \item Soit $\phi:\mathbb{R}^{2}\isoto\mathbb{R}^{2}$ définie par $\phi(x,y)=(ax,by)$. Alors l'image du cercle unité par $\phi$ est l'ellipse d'équation $\big(\frac{x}{a}\big)^{2}+\big(\frac{y}{b}\big)^{2}=1$.
  \item Soit $\phi:\mathbb{R}^{2}\isoto\mathbb{R}^{2}$ définie par $\phi(x,y)=(x-y,x+y)$.
    \begin{enumerate}[<+(1)->]
      \item $\phi(\{x^{2}+y^{2}=1\}) = \{x^{2}+y^{2}=2\}$.
      \item $\phi(\{ax+by=1\}) = \{x(a-b)+y(a+b)=2\}$.
    \end{enumerate}\pause
  \item Soit $\phi:\mathbb{R}^{3}\isoto\mathbb{R}^{3}$ définie par $\phi(x,y,z)=(x,z-y,z+y)$.\pause{} Alors l'image du cône standard est le cône d'équation $\{x^{2}=yz\}$.
  \end{enumerate}
\end{frame}

% ~~~~~~~~~~~~~~~~~~~~~~~~~~~~~~~~~~~~~~~
\section{Ellipse}
% ~~~~~~~~~~~~~~~~~~~~~~~~~~~~~~~~~~~~~~~

% -----------
\begin{frame}{Définition d'une ellipse}
  \begin{defprop}
    Soit $\ens{E}$ un plan affine euclidien. Un sous-ensemble $E$ est dit une \alert{ellipse} s'il satisfait une des conditions équivalentes:
    \begin{enumerate}[<+(1)->]
      \item Dans un repère orthonormé, $E$ est défini par une équation cartésienne de la forme $\big(\frac{x}{a}\big)^{2}+\big(\frac{y}{b}\big)^{2}=1$, où les constantes $a \geq b > 0$ sont appelées \alert{les rayons}.
      \item Il existe deux points $F_{1}$ et $F_{2}$ à distance $\d(F_{1},F_{2})=2c$, appelés \alert{foyers},  et $a > c$ tels que $E = \{ M \in \ens{E} \,|\, \d(F_{1},M)+\d(M,F_{2})=2a\}$.
      \item Soit $E$ est un cercle, soit il existe une droite $\ens{D}$, appelée \alert{la directrice}, un point $F \notin \ens{D}$, appelé \alert{foyer}, et $\varepsilon \in ]0,1[$, appelé \alert{excentricité}, tels que $E = \{ M \in \ens{E} \,|\, \d(F,M)=\varepsilon\,\d(M,\ens{D})\}$.
    \end{enumerate}
  \end{defprop}\pause
  D'autres définitions équivalentes d'une ellipse existent.\pause{}\newline
  Si $\d(F,\ens{D})=h$, alors on a les relations entre les paramètres : $a^{2}=c^{2}+b^{2}$, $\varepsilon^{2} =1-\big(\frac{b}{a}\big)^{2}$, et $(a-c)=\frac{\varepsilon}{1+\varepsilon}h$.
\end{frame}
% -----------
\begin{frame}{Propriétés des ellipses}
  \begin{enumerate}[<+(1)->]
    \item Toute ellipse possède un unique point de symétrie centrale, appelé \alert{le centre} de l'ellipse.
    \item Toute ellipse qui n'est pas un cercle admet exactement $2$ axes de symétrie orthogonaux qui se coupent dans le centre.
    \item Les cercles ont une infinité d'axes de symétrie: toute droite qui passe par le centre en est un.
    \item L'image par un isomorphisme (resp. similitude, resp. isométrie) affine d'une ellipse est une ellipse (resp. de même excentricité, resp. de mêmes rayons).
  \end{enumerate}
\end{frame}

% ~~~~~~~~~~~~~~~~~~~~~~~~~~~~~~~~~~~~~~~
\section{Parabole}
% ~~~~~~~~~~~~~~~~~~~~~~~~~~~~~~~~~~~~~~~

% -----------
\begin{frame}{Définition d'une parabole}
  \begin{defprop}
    Soit $\ens{E}$ un plan affine euclidien. Un sous-ensemble $P$ est dit une \alert{parabole} s'il satisfait une des conditions équivalentes:
    \begin{enumerate}[<+(1)->]
      \item Dans un repère orthonormé, $P$ est défini par une équation cartésienne de la forme $x^{2}=2py$ avec $p>0$.
      \item Il existe une droite $\ens{D}$, appelée \alert{la directrice}, et un point $F \notin \ens{D}$, appelé le \alert{foyer}, tels que $P = \{ M \in \ens{E} \,|\, \d(F,M)=\d(M,\ens{D})\}$.
    \end{enumerate}
  \end{defprop}\pause
  Les deux conditions sont reliées par l'équation $\d(F,\ens{D})=p$.\pause{}\newline
  Au vu de la deuxième condition, on dit que l'excentricité d'une parabole est $\varepsilon=1$.
\end{frame}
% -----------
\begin{frame}{Propriétés des paraboles}
  \begin{enumerate}[<+(1)->]
    \item Les paraboles n'ont pas de centre de symétrie.
    \item Toute parabole admet un unique axe de symétrie. Il est orthogonal à la directrice et passe par le foyer.
    \item L'image par un isomorphisme affine d'une parabole est une parabole.
  \end{enumerate}
\end{frame}

% ~~~~~~~~~~~~~~~~~~~~~~~~~~~~~~~~~~~~~~~
\section{Hyperbole}
% ~~~~~~~~~~~~~~~~~~~~~~~~~~~~~~~~~~~~~~~

% -----------
\begin{frame}{Définition d'une hyperbole}
  \begin{defprop}
    Soit $\ens{E}$ un plan affine euclidien. Un sous-ensemble $H$ est dit une \alert{hyperbole} s'il satisfait une des conditions équivalentes:
    \begin{enumerate}[<+(1)->]
      \item Dans un repère orthonormé, $H$ est défini par une équation cartésienne de la forme $\big(\frac{x}{a}\big)^{2}-\big(\frac{y}{b}\big)^{2}=1$ avec $a>0$ et $b>0$.
      \item Il existe deux points $F_{1}$ et $F_{2}$ à distance $\d(F_{1},F_{2})=2c$, appelés \alert{foyers},  et $a \in ]0,c\,[$ tels que $H = \big\{ M \in \ens{E} \,\big|\; |\!\d(F_{1},M)-\d(M,F_{2})|=2a\big\}$.
      \item Il existe une droite $\ens{D}$, appelée la \alert{la directrice}, un point $F \notin \ens{D}$, appelé \alert{foyer}, et $\varepsilon \in ]1,\infty[$, appelée \alert{excentricité}, tels que $H = \{ M \in \ens{E} \,|\, \d(F,M)=\varepsilon\,\d(M,\ens{D})\}$.
    \end{enumerate}
  \end{defprop}\pause
  D'autres définitions équivalentes d'une hyperbole existent.\pause{}\newline
  Si $\d(F,\ens{D})=h$, alors on a les relations entre les paramètres : $c^{2}=a^{2}+b^{2}$, $\varepsilon^{2} =1+\big(\frac{b}{a}\big)^{2}$, et $(c-a)=\frac{\varepsilon}{1+\varepsilon}h$.
\end{frame}
% -----------
\begin{frame}{Propriétés des hyperboles}
  \begin{enumerate}[<+(1)->]
    \item Il existe un repère (pas forcement orthonormé, ni même orthogonal) dans lequel l'équation de l'hyperbole est $xy=1$. Autrement dit c'est la graphe de la fonction $x\mapsto\frac{1}{x}$.
    \item Toute hyperbole possède un unique point de symétrie centrale, appelé \alert{le centre} de l'hyperbole.
    \item Toute hyperbole admet $2$ axes de symétrie orthogonaux qui se coupent dans le centre.
    \item Les deux droites d'équations $\frac{x}{a}\pm\frac{y}{b}=0$ sont appelées \alert{les asymptotes} de l'hyperbole $\big\{\big(\frac{x}{a}\big)^{2}-\big(\frac{y}{b}\big)^{2}=1\big\}$.
    \item L'image par un isomorphisme (resp. similitude, resp. isométrie) affine d'une hyperbole est une hyperbole (resp. de même excentricité, resp. de mêmes excentricité et distance entre les foyers).
  \end{enumerate}
\end{frame}

% ~~~~~~~~~~~~~~~~~~~~~~~~~~~~~~~~~~~~~~~
\section{Niveaux des formes quadratiques}
% ~~~~~~~~~~~~~~~~~~~~~~~~~~~~~~~~~~~~~~~

% -----------
\begin{frame}{Niveaux des formes quadratiques}
  Étant donnée une forme quadratique $Q$ de rang $r$ sur un espace vectoriel euclidien $\ev{E}$ de dimension $n$\pause, il existe une b.o.n dans laquelle $Q(\vv{v})=\lambda_{1}x_{1}^{2}+\cdots+\lambda_{r}x_{r}^{2}$\pause, où $(x_{1},\ldots,x_{n})$ sont les coordonnées de $\vv{v}$ dans cette b.o.n.\pause, et $\lambda_{i} \neq 0, \forall i=1,\ldots,r$.\pause

  Si on note $Q\,'$ la restriction de $Q$ sur l'espace $\ev{F}$ engendré par les $r$ premiers vecteurs de cette b.o.n., alors
  \begin{itemize}[<+(1)->]
    \item $\ev{F} = (\ker Q)^{\perp}$;
    \item $Q\,'$ est non dégénéré sur $\ev{F}$;
    \item La courbe de niveau $\ens{L}_{k}(Q)$ est un cylindre de direction $\ker Q$ et de base $\ens{L}_{k}(Q\,') \subset \ev{F}$.
  \end{itemize}\pause

  Donc pour étudier les courbes de niveaux des formes quadratiques, il suffit de connaître les courbes de niveaux des formes quadratiques \textbf{non dégénérées}.

\end{frame}

% -----------
\begin{frame}{En dimension 1}
  \begin{itemize}[<+(1)->]
    \item Toute forme quadratique non dégénérée sur $\mathbb{R}$ est de la forme $x\mapsto a x^{2}$ avec $a \neq 0$, et donc les courbes de niveaux $\ens{L}_{k}$ sont:
    \begin{itemize}[<+(1)->]
      \item \myemph{si $\signe(a)\neq\signe(k)$:} vides,
      \item \myemph{si $k=0$:} le point $\{0\}$,
      \item \myemph{si $\signe(a)=\signe(k)$:} l'ensemble de deux points $\{\pm\sqrt{k/a}\}$.
    \end{itemize}
    \item La seule forme quadratique dégénérée sur $\mathbb{R}$ est $x\mapsto 0$ dont les lignes de niveaux $\ens{L}_{k}$ sont:
    \begin{itemize}[<+(1)->]
      \item \myemph{si $k \neq 0$:} vides,
      \item \myemph{si $k=0$:} $\mathbb{R}$  qui est un cylindre de base $0$ et de direction $\mathbb{R}$.
    \end{itemize}
  \end{itemize}
\end{frame}

% -----------
\begin{frame}{En dimension 2}
  \begin{itemize}[<+(1)->]
  \item Les formes quadratiques non dégénérées de $\mathbb{R}^{2}$ s'écrivent dans une b.o.n. sous la forme $(x,y)\mapsto \lambda_{1}x^{2}+\lambda_{2}y^{2}$ avec $\lambda_{i}\neq0$ pour $i=1,2$.
    \begin{itemize}[<+(1)->]
      \item Si $\signe(\lambda_{1})=\signe(\lambda_{2})$ les courbes de niveaux $\ens{L}_{k}$ sont:
      \begin{itemize}[<+(1)->]
        \item \myemph{si $\signe(\lambda_{i})\neq\signe(k)$:} vide,
        \item \myemph{si $k=0$:} le point $\{0\}$,
        \item \myemph{si $\signe(\lambda_{i})=\signe(k)$:} une ellipse d'équation $\big(\frac{x}{a}\big)^{2}+\big(\frac{y}{b}\big)^{2}=1$ où $a=\sqrt{k/\lambda_{1}}$ et $b=\sqrt{k/\lambda_{2}}$.
      \end{itemize}
      \item Si $\signe(\lambda_{1})\neq\signe(\lambda_{2})$ les courbes de niveaux $\ens{L}_{k}$ sont:
      \begin{itemize}[<+(1)->]
        \item \myemph{si $k=0$:} deux droites de pentes $\pm\sqrt{-\lambda_{1}/\lambda_{2}}$,
        \item \myemph{si $k\neq0$:} une hyperbole dont l'équation standard dans le cas $k>0$ est $\big(\frac{x}{a}\big)^{2}-\big(\frac{y}{b}\big)^{2}=1$ où $a=\sqrt{k/\lambda_{1}}$ et $b=\sqrt{-k/\lambda_{2}}$.
      \end{itemize}\pause
      \myemph{Les asymptotes de l'hyperbole $\ens{L}_{k}$ sont les deux droites de $\ens{L}_{0}$.}
    \end{itemize}
  \item Les formes quadratiques dégénérées de $\mathbb{R}^{2}$ non nulles s'écrivent dans une b.o.n. sous la forme $(x,y)\mapsto \lambda x^{2}$ avec $\lambda\neq0$.\pause{} Et donc leurs courbes de niveaux $\ens{L}_{k}$ sont vides, une ou deux droites selon que $k/\lambda<0$, $k=0$ ou $k/\lambda>0$.
  \end{itemize}
\end{frame}

% -----------
\begin{frame}{En dimension 3}
  \begin{itemize}[<+(1)->]
  \item Les formes quadratiques non dégénérées de $\mathbb{R}^{3}$ s'écrivent dans une b.o.n. sous la forme $(x,y,z)\mapsto \lambda_{1}x^{2}+\lambda_{2}y^{2}+\lambda_{3}z^{2}$ avec $\lambda_{i}\neq0$ pour $i=1,2,3$.\pause{} Comme $\ens{L}_{k}(Q)=\ens{L}_{-k}(-Q)$ on peut restreindre notre étude au cas de $2$ ou $3$ valeurs propres positives\pause, et quitte à échanger les coordonnées on peut supposer $\lambda_{1},\lambda_{2} > 0$.
    \begin{itemize}[<+(1)->]
      \item Si $\lambda_{i}>0$ les courbes de niveaux $\ens{L}_{k}$ sont:
      \begin{itemize}[<+(1)->]
        \item \myemph{si $k<0$:} vide,
        \item \myemph{si $k=0$:} le point $\{0\}$,
        \item \myemph{si $k>0$:} un ellipsoïde d'équation $\big(\frac{x}{a}\big)^{2}+\big(\frac{y}{b}\big)^{2}+\big(\frac{z}{c}\big)^{2}=1$ où $a=\sqrt{k/\lambda_{1}}$, $b=\sqrt{k/\lambda_{2}}$ et $c=\sqrt{k/\lambda_{3}}$.
      \end{itemize}
      \item Si $\lambda_{1},\lambda_{2} > 0$ et $\lambda_{3} < 0$ les courbes de niveaux $\ens{L}_{k}$ sont:
      \begin{itemize}[<+(1)->]
        \item \myemph{si $k=0$:} un cône elliptique de base l'ellipse d'équations $\lambda_{1}x^{2}+\lambda_{2}y^{2}=|\lambda_{3}|$ et $z=1$,
        \item \myemph{si $k>0$:} un hyperboloïde à une nappe dont la section avec le plan $z=h$ est l'ellipse $\lambda_{1}x^{2}+\lambda_{2}y^{2}=k+|\lambda_{3}|h^{2}$.
        \item \myemph{si $k < 0$:} un hyperboloïde à deux nappes\pause{} dont la section avec le plan $z=h$ est vide si $|h|<\sqrt{k/\lambda_{3}}$\pause{} et est une l'ellipse $\lambda_{1}x^{2}+\lambda_{2}y^{2}=k+|\lambda_{3}|h^{2}$ si $|h|\geq\sqrt{k/\lambda_{3}}$.
      \end{itemize}
    \end{itemize}
  \end{itemize}
\end{frame}

% ~~~~~~~~~~~~~~~~~~~~~~~~~~~~~~~~~~~~~~~
\section{Coniques et quadriques}
% ~~~~~~~~~~~~~~~~~~~~~~~~~~~~~~~~~~~~~~~

% -----------
\begin{frame}{Fonctions polynomiales}
  \begin{defprop}
    On dit que $f$ est une \alert{fonction polynomiale} de degré $n$ sur l'espace affine $\ens{E}$ si dans un repère (dans tout repère) la valeur de $f(x_{1},\ldots,x_{n})$ est un polynôme de degré $n$ en les coordonnées $(x_{1},\ldots,x_{n})$.
  \end{defprop}\pause
  Dans le suite de ce cours on va étudier les courbes de niveaux des fonctions polynomiales de degré 2.
\end{frame}

% -----------
\begin{frame}{Pourquoi les polynômes de degré 2 ?}
  \begin{enumerate}[<+(1)->]
    \item Les courbes de niveaux des polynômes de degré 0 sont : vide ou tout l'espace.
    \item Les courbes de niveaux des polynômes de degré 1 sont les hyperplans affines.
    \item On vient de discuter les courbes de niveaux des polynômes homogènes de degré 2, qui sont les courbes de niveaux des formes quadratiques.
    \item Une fonction qui est un polynôme homogène de degré 2 dans un repère n'est pas forcement homogène dans un autre repère.
    \item On ne retrouve pas la parabole si on se limite aux polynômes homogènes.
  \end{enumerate}\pause
  Pour toutes ces raisons, il est naturel d'étudier en toute généralité les courbes de niveaux des fonctions polynomiales de degré 2 d'un espace affine (euclidien) $\ens{E}$ dans $\mathbb{R}$.
\end{frame}

% -----------
\begin{frame}{Forme standard des polynômes de degré 2}
  \begin{proposition}
    Soit $f$ est une fonction polynomiale de degré $2$ sur l'espace affine $\ens{E}$ de dimension $n$, alors il existe une b.o.n. dans laquelle soit
    \[
      f(x_{1},\ldots,x_{n}) = \lambda_{1}x_{1}^{2}+\cdots+\lambda_{r}x_{r}^{2}+\mu
    \]
    avec $\lambda_{i}\neq 0$ pour $i=1,\ldots,r$\pause, soit
    \[
      f(x_{1},\ldots,x_{n}) = \lambda_{1}x_{1}^{2}+\cdots+\lambda_{r}x_{r}^{2}+\mu x_{r+1}
    \]
     avec $\mu\neq0$ et $\lambda_{i}\neq 0$ pour $i=1,\ldots,r$.
  \end{proposition}\pause
  Dans le premier cas on se retrouve dans le cas déjà discuté des formes quadratiques.\pause\newline
  Dans le deuxième cas, en notant $\ens{F}=\{(x_{1},\ldots,x_{r+1},0,\ldots,0)\}$ et $\ens{F}^{\perp}=\{(0,\ldots,0,x_{r+2},\ldots,x_{n})\}$, vu que $\ens{L}_{k}(f\,)$ est un cylindre de direction $\ev{F^{\perp}}$ et de base $\ens{L}_{k}(f\,|_{\ens{F}}) \subset \ens{F}$, on peut se restreindre à l'étude du cas où $n=r+1$.
\end{frame}

% -----------
\begin{frame}{polynômes de degré 2, en dimension 1}
  En dimension $1$ il n'y a pas de polynômes de degré $2$ qui ne soient pas dans une base de la forme déjà étudiée $AX^{\,2}-C$.\pause\newline
  Calcul avec $a\neq0$ :
  \[
    ax^{2}+bx+c=a(x+\frac{b}{2a})^{2}-\frac{b^{2}}{4a}+c=AX^{\,2}-C,
  \]
  avec $A=a$, $X=x+\frac{b}{2a}$ et $C=\frac{b^{2}-4ac}{4a}$.
\end{frame}

% -----------
\begin{frame}{polynômes de degré 2, en dimension 2}
  En dimension $2$ les polynômes qui manquent dans notre étude s'écrivent dans une b.o.n. sous la forme $ax^{2}+by$ avec $a\neq0$ et $b\neq0$.\pause\newline
  Donc leurs courbes de niveaux $\ens{L}_{k}$ sont des paraboles $X^{\,2}=2pY$, avec $X=x$, $p=-\frac{b}{2a}$ et $Y=y-\frac{k}{b}$.
\end{frame}

% -----------
\begin{frame}{polynômes de degré 2, en dimension 3}
  En dimension $3$ il y a deux types de polynômes qu'il faut étudier en plus des formes quadratiques déjà étudiées :
  \begin{enumerate}[<+(1)->]
    \item $ax^{2}+by^{2}+cz$, avec $a$ et $b$ de même signe.\pause{} Ces courbes de niveaux sont appelées des \alert{paraboloïdes elliptiques}.\pause\newline
    Leur intersection avec un plan $\{z=h\}$ est une ellipse\pause, un point ou vide.\pause\newline
    Leur intersection avec un plan $\{x=h\}$ ou $\{y=h\}$ est une parabole.
    \item $ax^{2}-by^{2}+cz$, avec $a>0$ et $b>0$. \pause{} Ces courbes de niveaux sont appelées des \alert{paraboloïdes hyperboliques}.\pause\newline
    Leur intersection avec un plan $\{z=h\}$ est une hyperbole\pause{} ou deux droites.\pause\newline
    Leur intersection avec un plan $\{x=h\}$ ou $\{y=h\}$ est une parabole.
  \end{enumerate}
\end{frame}

\end{document}
