% \PassOptionsToClass{handout}{m53beamer}
\documentclass{m53beamer}
% ---------------
\title{M53\\Géométrie affine et euclidienne}
\date{septembre 2017}
% ---------------

\begin{document}

% ------- Le titre --------
\begin{frame}
  \titlepage
\end{frame}

% =======================================
\section{Informations utiles}
% =======================================
% -----------
\begin{frame}{Informations utiles}
  \begin{enumerate}[<+(1)->]
    \item Chargés de cours/TDs :
          \begin{itemize}
            \item \textbf{Kroum Tzanev} (cours)\\
              bât. M2, bur.202, mél:~\texttt{kroum.tzanev@math.univ-lille1.fr}
            \item \textbf{Patrick Popescu-Pampu} (td goupe 1)\\
              bât. M3, bur.211, mél:~\texttt{patrick.popescu@math.univ-lille1.fr}
            \item \textbf{Changgui Zhang} (td goupe 2)\\
              bât. M2, bur.103, mél:~\texttt{changgui.zhang@math.univ-lille1.fr}
            \item \textbf{Marc Bourdon} (td goupe 3)\\
              bât. M3, bur.229, mél:~\texttt{marc.bourdon@math.univ-lille1.fr}
          \end{itemize}
    \item Ce module se déroule sur 10 semaines (20h de cours et 30h de tds).
  \end{enumerate}
\end{frame}

% =======================================
\section{Resources}
% =======================================
% -----------
\begin{frame}{Resources}
  \begin{enumerate}[<+(1)->]
    \item Il faut s'inscrire sur Moodle (pas de clé).
    \item Le cours et les tds sont disponibles à l'adresse :
      \texttt{https://ktzanev.github.io/m53lille1}
    \item L'ouvrage principal est le livre «Géométrie» de Michèle Audin.
    \item On trouve d'autres cours sur internet en cherchant «géométrie affine cours pdf».
  \end{enumerate}
\end{frame}

% =======================================
\section{Évaluation}
% =======================================
% -----------
\begin{frame}{Évaluation}
  \begin{enumerate}[<+(1)->]
    \item Les contrôles sont :
          \begin{itemize}
            \item Un interrogation écrite de 1h (note $I$).
            \item DS1 de 2h (note $DS_{1}$).
            \item DS2 de 3h (note $DS_{2}$).
            \item Rattrapage 3h (la note remplace $DS_{2}$).
          \end{itemize}
    \item La note finale est
      \[
        \max\left\{\dfrac{I + 2DS_{1} + 3DS_{2}}{6},DS_{2}\right\}.
      \]
  \end{enumerate}
\end{frame}
\end{document}
