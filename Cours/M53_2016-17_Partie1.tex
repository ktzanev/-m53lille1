%\PassOptionsToClass{negatif}{m53beamer}
%\PassOptionsToClass{withsidebar}{m53beamer}
%\PassOptionsToClass{handout}{m53beamer}
%\PassOptionsToClass{print}{m53beamer}
\documentclass{m53beamer}
% ---------------
\title{M53 - Partie 1}
%\author{Kroum Tzanev}
\date{septembre 2016}
% ---------------

\begin{document}

% ------- Le titre --------
\begin{frame}
  \titlepage
\end{frame}
% =======================================
\section{« Rappels » sur la géométrie plane}
% =======================================

% ~~~~~~~~~~~~~~~~~~~~~~~~~~~~~~~~~~~~~~~
\subsection{Le plan réel}
% ~~~~~~~~~~~~~~~~~~~~~~~~~~~~~~~~~~~~~~~

% -----------
\begin{frame}{Le plan réel}
  \begin{itemize}[<+(1)->]
    \item Un ensemble $\ens{P}$ peut être considéré comme un \myemph{plan} s'il est \frquote{naturellement} isomorphe à $\mathbb{R}^{2}$. \pause Dans ce cas, les éléments de $\ens{P}$ sont appelés des \myemph{points} ou des \myemph{vecteurs} en fonction du contexte.
    \item Si on fixe un isomorphisme entre un plan $\ens{P}$ et $\mathbb{R}^{2}$, à tout point (vecteur) $M \in \ens{P}$ on fait correspondre un couple de nombres $(x,y)$ appelé \myemph{coordonnées (cartésiennes)} de $M$.
    \item Ainsi toute notion de $\mathbb{R}^{2}$ peut être \frquote{transportée} à $\ens{P}$ : droites, cercles, distances, produit scalaire, angles, \ldots
    \item L'application $(\rho,\theta) \mapsto (\rho\cos(\theta),\rho\sin(\theta))$ est une surjection de $\mathbb{R}_{+}\times\mathbb{R}$ sur $\mathbb{R}^{2}$. \pause Ainsi la donnée d'un couple $(\rho,\theta)$, appelé \myemph{coordonnées polaires}, détermine un unique point de $\mathbb{R}^{2}$ (et ainsi éventuellement de $\ens{P}$).
  \end{itemize}
\end{frame}

% ~~~~~~~~~~~~~~~~~~~~~~~~~~~~~~~~~~~~~~~
\subsection{Les droites de \texorpdfstring{$\mathbb{R}^{2}$}{R²}}
% ~~~~~~~~~~~~~~~~~~~~~~~~~~~~~~~~~~~~~~~

% -----------
\begin{frame}{Les droites de \texorpdfstring{$\mathbb{R}^{2}$}{R²}}
  \begin{itemize}[<+(1)->]
    \item Une droite (affine) de $\mathbb{R}^{2}$ est un ensemble défini par une équation de la forme $ax+by=d$ avec $(a,b) \neq (0,0)$.
    \item Une telle droite est vectorielle ssi $d=0$.
    \item Une droite n'a pas une équation unique : toutes les équations qui définissent la même droite que $ax+by=d$ sont de la forme $\lambda ax+\lambda by=\lambda d$ avec $\lambda\neq0$.
    \item Le vecteur $(a,b)$ est \alert{normal} à la droite définie par $ax+by=d$.
    \item On dit que l'équation $ax+by=d$ est \myemph{normalisée} si $\norm{(a,b)}=1$. La droite définie par une telle équation normalisée est à distance $|d|$ de $0$.
    \item Deux droites définies par $a_{1}x+b_{1}y=d_{1}$ et $a_{2}x+b_{2}y=d_{2}$ sont parallèles ssi les vecteurs $(a_{1},b_{1})$ et $(a_{2},b_{2})$ sont colinéaires.
    \item Toutes les droites parallèles à une droite $ax+by=d$ admettent une équation de la forme $ax+by=\delta$ ($\delta \in \mathbb{R}$).
  \end{itemize}
\end{frame}

% ~~~~~~~~~~~~~~~~~~~~~~~~~~~~~~~~~~~~~~~
\subsection{Le plan complexe}
% ~~~~~~~~~~~~~~~~~~~~~~~~~~~~~~~~~~~~~~~

% -----------
\begin{frame}{Le plan complexe}
  \begin{itemize}[<+(1)->]
    \item Comme $\mathbb{C}$ est \frquote{naturellement} isomorphe à $\mathbb{R}^{2}$ on peut le considérer comme un plan (appelé \myemph{le plan complexe}).
    \item Ainsi à tout point (vecteur) de $\mathbb{R}^{2}$ on peut faire correspondre un nombre complexe appelé son \myemph{affixe}.
    \item La formule d'Euler $e^{i\theta}=\cos(\theta)+i\sin(\theta)$ nous permet d'identifier les coordonnées polaires $(\rho,\theta)$ d'un point avec le module et l'argument de son affixe.
    \item Le produit scalaire de deux vecteurs $\vv{v_{1}}$ et $\vv{v_{2}}$ d'affixes respectives $z_{1}$ et $z_{2}$ s'écrit
    \[
        \scalprod{v_{1}}{v_{2}} = \frac{z_{1}\overline{z_{2}}+\overline{z_{1}}z_{2}}{2}\,.
    \]
    \item Une grande partie des opérations algébriques sur $\mathbb{C}$ ont une interprétation géométrique.
  \end{itemize}
\end{frame}


% =======================================
\section{Espace affine}
% =======================================
% ~~~~~~~~~~~~~~~~~~~~~~~~~~~~~~~~~~~~~~~
\subsection{Définition}
% ~~~~~~~~~~~~~~~~~~~~~~~~~~~~~~~~~~~~~~~
% -----------
\begin{frame}{La définition d'un espace affine}
  \begin{definition}[heuristique]
    \frquote{Un espace affine est un espace vectoriel dont on a oublié l'origine.}
  \end{definition}
\end{frame}
% -----------
\begin{frame}{La définition d'un espace affine}
  \begin{definition}
    Soit $\ev{E}$ un espace vectoriel \emph{(si non précisé, sur $\mathbb{R}$)}.\pause\\
    Un ensemble (non vide) $\ens{E}$ est muni de la structure d'\alert{espace affine de direction $\ev{E}$}
    par la donnée d'une application
    \begin{gather*}
      \ens{E} \times \ens{E} \longrightarrow \ev{E}\\
      (A,B) \mapsto \vv{AB}
    \end{gather*}\pause
    qui satisfait les deux conditions:
    \begin{enumerate}[<+(1)->]
      \item $\vv{AB}+\vv{BC}=\vv{AC}$ \myemph{(relation de Chasles)}
      \item $\forall A \in \ens{E}, \vv{v} \in \ev{E}$\pause , $\exists ! B \in \ens{E}$ t.q. $\vv{AB}=\vv{v}$
      \myemph{($B=A+\vv{v}$)}
    \end{enumerate}
  \end{definition}
\end{frame}
% -----------
\begin{frame}{La dimension d'un espace affine}
  Si $\ens{E}$ est un espace affine sur $\ev{E}$ et $\phi:\ev{E}\isoto\ev{F}$ un isomorphisme d'espaces vectoriels, alors la composition
    \begin{gather*}
      \ens{E} \times \ens{E} \longrightarrow \ev{E}\isoto\ev{F}\\
      (A,B) \mapsto \phi(\vv{AB})
    \end{gather*}
  définit une structure d'espace affine sur $\ens{E}$ de direction $\ev{F}$. \pause On parle dans ce cas de \myemph{changement de direction} de $\ens{E}$.
  \pause
  \begin{definition}
    L'espace affine $\ens{E}$ est de dimension $n$ si sa direction, l'espace vectoriel $\ev{E}$, est de dimension $n$.
  \end{definition}
\end{frame}
% ~~~~~~~~~~~~~~~~~~~~~~~~~~~~~~~~~~~~~~~
\subsection{Exemples}
% ~~~~~~~~~~~~~~~~~~~~~~~~~~~~~~~~~~~~~~~
  % -----------
  \begin{frame}{Les espaces vectoriels}
    Tout espace vectoriel $\ev{E}$ peut être muni naturellement d'une structure d'espace affine, avec direction lui-même, via l'application :
    \begin{gather*}
      \ev{E} \times \ev{E} \longrightarrow \ev{E}\\
      (\vv{A},\vv{B}) \mapsto \vv{AB}=\vv{B}-\vv{A}
    \end{gather*}\pause
    \begin{block}{Convention}
      Dans la suite, tous les espaces vectoriels vont être considérés munis de cette structure naturelle d'espace affine.
    \end{block}
  \end{frame}
  % -----------
  \begin{frame}{Les droites (sous-espaces) affines}
    Le sous-ensemble de $\mathbb{R}^{2}$, $\ens{E}=\{(x,y)\ |\ x+y=1 \}$ est un espace affine de direction $\ev{E}=\{(x,y)\ |\ x+y=0 \}$, via l'application
    \begin{gather*}
      \ens{E} \times \ens{E} \longrightarrow \ev{E}\\
      (A,B) \mapsto \vv{AB}=B-A
    \end{gather*}\pause
    \begin{question}
      Comment peut-on généraliser cet exemple ?
    \end{question}
  \end{frame}
  % -----------
  \begin{frame}{Les solutions des équations différentiels linéaires}
    L'ensemble des solutions $S$ de l'équation différentielle $y\,'+y=sin(x)$ est un espace affine avec direction $S^{\,\ast}$, l'ensemble des solutions de l'équation homogène ($y\,'+y=0$) via :
    \begin{gather*}
      S \times S \longrightarrow S^{\,\ast}\\
      (f_{1},f_{2}) \mapsto f_{2}-f_{1}
    \end{gather*}\pause
    \begin{question}
      Comment peut-on généraliser cet exemple ?
    \end{question}
  \end{frame}
% ~~~~~~~~~~~~~~~~~~~~~~~~~~~~~~~~~~~~~~~
\subsection{Opérations}
% ~~~~~~~~~~~~~~~~~~~~~~~~~~~~~~~~~~~~~~~
  % -----------
  \begin{frame}{Vectorialisé d'un espace affine}
    En fixant un point $\Omega$ d'un espace affine $\ens{E}$, on peut définir sur celui-ci une structure d'espace vectoriel via la bijection:
    \begin{gather*}
      \ens{E} \longrightarrow \ev{E}\\
      B \mapsto \vv{\Omega B}
    \end{gather*}\pause
    Cet espace vectoriel est noté $\ens{E}_{\Omega}$ et est isomorphe (par définition) à $\ev{E}$.
    \begin{enumerate}[<+(1)->]
      \item L'origine de $\ens{E}_{\Omega}$ est le point $\Omega$.
      \item Avec l'écriture $\Omega+\vv{v}$, les opérations sont:
      \begin{itemize}[<+(1)->]
        \item $(\Omega+\vv{v})+(\Omega+\vv{w})=(\Omega+\vv{v}+\vv{w})$.
        \item $\lambda(\Omega+\vv{v})=\Omega+\lambda\vv{v}$.
      \end{itemize}
    \end{enumerate}\pause
    \begin{remarque}
      On peut considérer $\ens{E}_{\Omega}$ comme espace affine avec direction lui même \pause ou avec direction $\ev{E}$\pause, la différence est le \frquote{changement de direction} via l'isomorphisme linéaire $\ens{E}_{\Omega}\isoto\ev{E}$.
    \end{remarque}
  \end{frame}
  % -----------
  \begin{frame}{Produit d'espaces affines}
    Soient $\ens{E}$ et $\ens{F}$ deux espaces affines, sur le même corps, de directions respectives $\ev{E}$ et $\ev{F}$.\\
    On définit la structure d'espace affine \myemph{produit} sur $\ens{E} \times \ens{F}$ de direction $\ev{E} \times \ev{F}$ par:
    $$
      \vv{(A,B)(C,D)}:=(\vv{AC},\vv{BD}).
    $$
  \end{frame}
% ~~~~~~~~~~~~~~~~~~~~~~~~~~~~~~~~~~~~~~~
\subsection{Premières propriétés}
% ~~~~~~~~~~~~~~~~~~~~~~~~~~~~~~~~~~~~~~~
  % -----------
  \begin{frame}{Propriétés calculatoires}
    Soit $\ens{E}$ un $\mathbb{K}$-espace affine de direction $\ev{E}$.
    \begin{enumerate}[<+(1)->]
      \item $A \in \ens{E} \quad\Rightarrow\quad \vv{AA}=\vv{0}$ et $A+\vv{0}=A$.
      \item $A,B \in \ens{E} \quad\Rightarrow\quad \vv{AB}=-\vv{BA}$.
      \item $A+\vv{v}=B  \quad\Leftrightarrow\quad \forall (\exists) C \in \ens{E}, \vv{CA}+\vv{v}=\vv{CB}$.
      \item $(A+\vv{v})+\vv{w}=A+(\vv{v}+\vv{w})$ \myemph{($\ev{E}$ agit sur $\ens{E}$)}.
      \item $\vv{AB}=\vv{DC} \Leftrightarrow \vv{AD}=\vv{BC}$ \myemph{($ABCD$ est un parallélogramme)}.
      \item $\vv{(A+\vv{v})(B+\vv{w})}=\vv{AB}-\vv{v}+\vv{w}$.
      \item Soient $A_{1},\ldots,A_{k} \in \ens{E}$ et $\lambda_{1},\ldots,\lambda_{k} \in \mathbb{K}$
        \begin{itemize}[<+(1)->]
          \item Si $\sum_{i=1}^{k}\lambda_{i}=0$ alors $\sum_{i=1}^{k}\lambda_{i}A_{i} \in \ev{E}$ est bien définie \myemph{($\vv{AB} = B - A$)}.
          \item Si $\sum_{i=1}^{k}\lambda_{i}=1$ alors $\sum_{i=1}^{k}\lambda_{i}A_{i} \in \ens{E}$ est bien définie.
          \item Si $\sum_{i=1}^{k}\lambda_{i}\notin \{0,1\}$ alors $\sum_{i=1}^{k}\lambda_{i}A_{i}$ \frquote{n'est pas bien définie}.
        \end{itemize}\pause
    \end{enumerate}
  \end{frame}
% =======================================
\section{Barycentre et repères}
% =======================================
% ~~~~~~~~~~~~~~~~~~~~~~~~~~~~~~~~~~~~~~~
\subsection{Barycentre}
% ~~~~~~~~~~~~~~~~~~~~~~~~~~~~~~~~~~~~~~~
  % -----------
  \begin{frame}{Définition du barycentre}
    \begin{defprop}
      Soient $A_{1},\ldots,A_{k} \in \ens{E}$ et $\mu_{1},\ldots,\mu_{k} \in \mathbb{K}$ tels que $\sum_{i=1}^{k}\mu_{i} \neq 0$, alors il existe un unique point $G$ qui satisfait une des conditions équivalentes:
        \begin{enumerate}[<+(1)->]
          \item $G = \sum_{i=1}^{k}\frac{\mu_{i}}{\sum_{i=1}^{k}\mu_{i}}A_{i}$.
          \item $\forall(\exists)M\in \ens{E},\ (\sum_{i=1}^{k}\mu_{i})\vv{MG}=\sum_{i=1}^{k}\mu_{i}\vv{MA_{i}}$.
          \item $\sum_{i=1}^{k}\mu_{i}\vv{GA_{i}} = 0$.
        \end{enumerate}
      Le point $G$ est \alert{le barycentre} des \alert{des points pondérées} $\{(A_{1},\mu_{1}),\ldots,(A_{k},\mu_{k})\}$, et les $\{\mu_{i}\}$ sont appelés les \alert{poids}.
    \end{defprop}\pause
    \begin{definition}
      Soient $A_{1},\ldots,A_{k} \in \ens{E}$, leur \alert{isobarycentre} est le barycentre de ces points pondérés du même poids non nul \emph{(qui peut être pris égal à $\frac{1}{k}$, ou à $1$)}.
    \end{definition}
  \end{frame}
% ~~~~~~~~~~~~~~~~~~~~~~~~~~~~~~~~~~~~~~~
\subsection{Propriétés}
% ~~~~~~~~~~~~~~~~~~~~~~~~~~~~~~~~~~~~~~~
  % -----------
  \begin{frame}{Propriétés des barycentres}
    \begin{enumerate}[<+(1)->]
      \item Si on remplace les poids $\mu_{i}$ par $\lambda\mu_{i}$ pour $\lambda \neq 0$, le barycentre ne change pas.
      \item Si on rajoute un point pondéré par un poids nul, le barycentre ne change pas.
      \item Soit $\ens{E} \times \ens{F}$ un espace affine produit.\\
      Le barycentre des points pondérés $\{\big((A_{1},B_{1}),\mu_{1}\big),\ldots,\big((A_{k},B_{k}),\mu_{k}\big)\}$ est $G=(G_{A},G_{B})$,\\
      où $G_{A}$ est le barycentre de $\{(A_{1},\mu_{1}\big),\ldots,(A_{k},\mu_{k})\}$ dans $\ens{E}$,\\
      et $G_{B}$ est le barycentre de $\{(B_{1},\mu_{1}),\ldots,(B_{k},\mu_{k})\}$ dans $\ens{F}$.
    \end{enumerate}
  \end{frame}
  % -----------
  \begin{frame}{Associativité du barycentre}
    Soient $\{A_{i}\}_{i \in I}$ des points de $\ens{E}$ et $\{\mu_{i}\}_{i \in I}$ des scalaires de somme non nulle, indexés par un  ensemble $I$.\pause\\
    Soit une partition $I =J_{1} \sqcup\dots\sqcup J_{r}$, telle que $\nu_{k}:=\sum_{i \in J_{k}}\mu_{i} \neq 0$ pour chaque $k \in \{1,\ldots,r\}$.\pause\\
    On note $G_{k}$ le barycentre de $\{(A_{i},\mu_{i})\}_{i \in J_{k}}$.\pause
    \begin{proposition}
      Le barycentre $G$ des points pondérés $\{(A_{i},\mu_{i})\}_{i \in I}$ est aussi le barycentre des $\{(G_{k},\nu_{k})\}_{k \in \{1,\ldots,r\}}$.
    \end{proposition}
  \end{frame}
% ~~~~~~~~~~~~~~~~~~~~~~~~~~~~~~~~~~~~~~~
\subsection{Repères}
% ~~~~~~~~~~~~~~~~~~~~~~~~~~~~~~~~~~~~~~~
  % -----------
  \begin{frame}{Repère cartésien}
    \begin{definition}
      Un \alert{repère cartésien} d'un espace affine $\ens{E}$ de dimension $n$ est la donnée $\mathcal{C}=(\Omega, \vv{v_{1}}, \ldots,\vv{v_{n}})$ d'un point $\Omega$ de $\ens{E}$, \myemph{  l'origine du repère}, et d'une base $(\vv{v_{1}}, \ldots,\vv{v_{n}})$ de la direction $\ev{E}$.
    \end{definition}\pause
    \begin{definition}
      Soit $\mathcal{C}=(\Omega, \vv{v_{1}}, \ldots,\vv{v_{n}})$ un repère cartésien de $\ens{E}$ et $M$ un point de $\ens{E}$. On dit que $(x_{1},\ldots,x_{n})$ sont les \alert{coordonnées cartésiennes} de $M$ dans le repère $\mathcal{C}$, et on note $M=(x_{1},\ldots,x_{n})_{\mathcal{C}}$, si ce sont les coordonnées de $\vv{\Omega M}$ dans la base $(\vv{v_{1}}, \ldots,\vv{v_{n}})$ de la direction $\ev{E}$.\pause\ Autrement dit $M=(x_{1},\ldots,x_{n})_{\mathcal{C}}$ si et seulement si $M=\Omega+\sum_{i=1}^{n}x_{i}\vv{v_{i}}$.
    \end{definition}
  \end{frame}
  % -----------
  \begin{frame}{Repère affine}
    \begin{definition}
      On dit que le $(n+1)$-uplet $(A_{0},\ldots,A_{n})$ est un \alert{repère affine} de $\ens{E}$ si pour tout point $M$ de $\ens{E}$ il existe un unique $(n+1)$-uplet de poids $(\mu_{0},\ldots,\mu_{n})$, avec $\sum_{i=0}^{n}\mu_{i}=1$ et $M=\sum_{i=1}^{n}\mu_{i}A_{i}$.
    \end{definition}\pause
    \begin{definition}
      Soit $\mathcal{A}=(A_{0},\ldots,A_{n})$ est un repère affine de $\ens{E}$. On dit que $(\mu_{0},\ldots,\mu_{n})$ sont les \alert{coordonnées barycentriques} dans le repère $\ens{A}$ d'un point $M$ de $\ens{E}$, et on note $M=[\mu_{0},\ldots,\mu_{n}]_{\ens{A}}$, si $\sum_{i=0}^{n}\mu_{i}=1$ et $M=\sum_{i=0}^{n}\mu_{i}A_{i}$.
    \end{definition}\pause
    \begin{remarque}
      Si $\mathcal{A}=(A_{0},\ldots,A_{n})$ est un repère affine de $\ens{E}$, alors toute permutation $\mathcal{A}=(A_{\sigma(0)},\ldots,A_{\sigma(n)})$ l'est aussi.
    \end{remarque}
  \end{frame}
  % -----------
  \begin{frame}{Relations entre repères cartésiens et affines}
    \begin{proposition}
      Le $(n+1)$-uplet $\mathcal{A}=(A_{0},\ldots,A_{n})$ est un \myemph{repère affine} de $\ens{E}$ si et seulement si $\mathcal{C}=(A_{0},\vv{A_{0}A_{1}}\ldots,\vv{A_{0}A_{n}})$ est un \myemph{repère cartésien} de $\ens{E}$.\pause\newline
      De plus si $M = (x_{1},\ldots,x_{n})_{\mathcal{C}}$ et $M = [\mu_{0},\ldots,\mu_{n}]_{\ens{A}}$ alors la relation entre ces deux systèmes de coordonnées est : $\mu_{i}=x_{i}, \forall i=1,\ldots,n$ et $\mu_{0}=1-\sum_{i=1}^{n}x_{i}$.
    \end{proposition}
  \end{frame}

% =======================================
\section{Sous-espaces affines}
% =======================================
% ~~~~~~~~~~~~~~~~~~~~~~~~~~~~~~~~~~~~~~~
\subsection{Définition}
% ~~~~~~~~~~~~~~~~~~~~~~~~~~~~~~~~~~~~~~~
  % -----------
  \begin{frame}{Définition d'un sous-espace affine}
      Soit $\ens{E}$ un espace affine.
      \begin{defprop}
      Un sous-ensemble non vide $\ens{F} \subset \ens{E}$ est dit \alert{sous-espace affine} s'il satisfait une des conditions équivalentes:
      \begin{enumerate}[<+(1)->]
        \item Il existe un sous-espace vectoriel $\ev{F}$ de $\ev{E}$ et $\Omega \in \ens{E}$ tels que $\ens{F}=\Omega+\ev{F}$.
        \item $\exists (\forall)\Omega \in \ens{F}$, $\ens{F}$ est un sous-espace vectoriel de $\ens{E}_{\Omega}$.
        \item $\ens{F}$ est stable par barycentres.
      \end{enumerate}
    \end{defprop}\pause
    Un sous-espace affine $\ens{F}=\Omega+\ev{F}$ est un espace affine de direction $\ev{F}$, via la restriction de l'application $(A,B)\mapsto\vv{AB}$.
  \end{frame}
% ~~~~~~~~~~~~~~~~~~~~~~~~~~~~~~~~~~~~~~~
\subsection{Exemples}
% ~~~~~~~~~~~~~~~~~~~~~~~~~~~~~~~~~~~~~~~
  % -----------
  \begin{frame}{Sous-espaces affines et dimensions}
    Soit $\ens{E}$ un espace affine de dimension $n$.
    \begin{enumerate}[<+(1)->]
      \item Les sous-espaces affines de dimension $0$ sont les points $\{M\}$ de $\ens{E}$.
      \item Les sous-espaces affines de dimension $1$ sont appelés \alert{des droites affines}.
      \item Les sous-espaces affines de dimension $2$ sont appelés \alert{des plans affines}.
      \item Les sous-espaces affines de dimension $n-1$ sont appelés \alert{des hyperplans affines}.
    \end{enumerate}\pause

    Soient $A$ et $B$ deux points distincts de $\ens{E}$. On note $AB$ ou $\affspan{A,B}$ la droite affine qui passe par $A$ et $B$.
  \end{frame}
  % -----------
  \begin{frame}{Les sous-espaces affines d'un espace vectoriel}
    Soient $\ev{E}$ et $\ev{G}$ deux espaces vectoriels, et $\vv\phi \in \lin(\ev{E},\ev{G})$ une application linéaire.\pause\\
    \begin{proposition}
      Pour tout $\vv{v}\in\im\vv\phi\subset\ev{G}$, l'image réciproque $\vv\phi^{-1}(\vv{v})$ est un sous-espace affine de $\ev{E}$ de direction $\ker{\vv\phi}$.
    \end{proposition}
    \begin{enumerate}[<+(1)->]
      \item En particulier, en prenant $\vv\phi(x,y)=x+y$ de $\mathbb{R}^{2}$ dans $\mathbb{R}$ et $\vv{v}=1$, on retrouve le sous-espace affine $\ens{E}=\{(x,y)\ |\ x+y=1 \}$ de direction $\ev{E}=\{(x,y)\ |\ x+y=0 \}$.
      \item L'ensemble $\ens{S}$ des solutions d'un système linéaire $AX=B$ est vide ou est un sous-espace affine de direction l'ensemble $\ens{S^{\,\ast}}$ des solutions homogènes $AX=0$. Et $\ens{S}=X_{0}+\ens{S}'$, où $X_{0}$ est une solution particulière.
    \end{enumerate}
  \end{frame}
  % -----------
  \begin{frame}{Les sous-espaces affines d'un espace vectoriel}
    Soient $\ev{E}$ un espace vectoriel et $\ens{F}$ un sous-espace affine de $\ev{E}$.
    \begin{enumerate}[<+(1)->]
      \item $\ens{F}$ est un sous-espace vectoriel ssi $0 \in \ens{F}$.
      \item $\ens{F}$ est un hyperplan affine ssi il existe une forme linéaire non nulle $\vv\phi\in\ev{E}^{\,*}$ et $a \in \mathbb{R}$, tels que $\ens{F}=\vv\phi^{-1}(a)$.
      \item Tous les sous-espaces affines de $\mathbb{R}^{n}$ sont des ensembles de solutions de systèmes linéaires.
    \end{enumerate}
  \end{frame}
% ~~~~~~~~~~~~~~~~~~~~~~~~~~~~~~~~~~~~~~~
\subsection{Propriétés}
% ~~~~~~~~~~~~~~~~~~~~~~~~~~~~~~~~~~~~~~~
  % -----------
  \begin{frame}{Parallélisme}
    \begin{definition}
      On dit que deux (plusieurs) sous-espaces affines d'un même espace affine sont
      \alert{parallèles} s'ils ont la même direction.\pause\myemph{(C'est une relation d'équivalence.)}
    \end{definition}\pause
    Attention : \frquote{disjoints} $\not\Rightarrow$ \frquote{parallèles}.\pause
    \begin{proposition}
      \begin{enumerate}[<+->]
        \item Deux sous-espaces parallèles sont disjoints ou confondus.
        \item Par tout point d'un espace affine, il passe une unique droite \uncover<+>{(sous-espace)} parallèle à une droite \uncover<.>{(sous-espace)} donnée.
      \end{enumerate}
    \end{proposition}
  \end{frame}
  % -----------
  \begin{frame}{Intersection de sous-espaces affines}
    \begin{proposition}
      L'intersection de deux sous-espaces affines $\ens{F}$ et $\ens{G}$ est:
      \begin{itemize}\pause
          \item vide\pause, ou
          \item un sous-espace affine de direction $\ev{F}\cap\ev{G}$.
        \end{itemize}
    \end{proposition}\pause
    \begin{proposition}
      L'intersection de deux sous-espaces affines $\ens{F}$ et $\ens{G}$ est vide si et seulement si $\exists(\forall) A \in \ens{F}, B \in \ens{G}$,
      $$
        \vv{AB} \notin \ev{F}+\ev{G}.
      $$
    \end{proposition}
  \end{frame}
  % -----------
  \begin{frame}{Sous-espace engendré}
    \begin{defprop}
      Soit $\ens{A}$ un sous-ensemble non vide d'un espace affine $\ens{E}$. Le sous-espace affine $\affspan{\ens{A}}$ engendré par $\ens{A}$ est défini par une des conditions équivalentes:
        \begin{enumerate}[<+(1)->]
          \item $\affspan{\ens{A}}$ est le plus petit sous-espace affine contenant $\ens{A}$.
          \item $\affspan{\ens{A}}$ est l'intersection de tous les sous-espaces affines contenant $\ens{A}$.
          \item $\affspan{\ens{A}}$ est l'ensemble des barycentres de points de $\ens{A}$.
          \item $\forall(\exists) \Omega \in \ens{A}$, $\affspan{\ens{A}}$ est le sous-espace vectoriel engendré par $\ens{A}$ dans $\ens{E}_{\Omega}$.
        \end{enumerate}
    \end{defprop}
  \end{frame}
  % -----------
  \begin{frame}{Somme de sous-espaces affines}
    \begin{proposition}
      Soient $\ens{F}$ et $\ens{G}$ deux sous-espaces affines du même espace affine, et $\affspan{\ens{F},\ens{G}}$ le sous-espace affine engendré par $\ens{F} \cup \ens{G}$.
      \begin{enumerate}[<+(1)->]
        \item Si $\ens{F} \cap \ens{G} \neq \emptyset$, alors $\affspan{\ens{F},\ens{G}}$ est de direction $\ev{F}+\ev{G}$\pause, et
            $$
              \dim\affspan{\ens{F},\ens{G}} = \dim \left(\ev{F}+\ev{G}\right).
            $$
        \item Si $\ens{F} \cap \ens{G} = \emptyset$, alors $\affspan{\ens{F},\ens{G}}$ est de direction $\ev{F}+\ev{G}+\vv{D}$, où $\vv{D}$ est une droite engendrée par $\vv{AB}$ avec $A \in \ens{F}$ et $B \in \ens{G}$\pause, et
            $$
              \dim\affspan{\ens{F},\ens{G}} = \dim \left(\ev{F}+\ev{G}\right)+1.
            $$
      \end{enumerate}
    \end{proposition}
  \end{frame}
  % -----------
  \begin{frame}{Familles affinement libres et génératrices}
    Soit $\ens{F}$ un sous-espace affine d'un espace affine $\ens{E}$.\pause
    \begin{definition}
      Soient $\{A_{0},\ldots,A_{k}\}$ des points de $\ens{F}$. On dit que cette famille est \alert{affinement génératrice} pour $\ens{F}$ si $\affspan{A_{0},\ldots,A_{k}}=\ens{F}$.
    \end{definition}\pause
    \begin{definition}
      Soient $(k+1)$ points $\{A_{0},\ldots,A_{k}\}$ de $\ens{E}$. On dit que cette famille est \alert{affinement libre} si $\dim\affspan{A_{0},\ldots,A_{k}}=k$.
    \end{definition}
  \end{frame}
  \begin{frame}{Caractérisation d'un repère}
    Soit $\ens{F}$ un sous-espace affine d'un espace affine $\ens{E}$.
    \begin{proposition}
      Le $(k+1)$-uplet $(A_{0},\ldots,A_{k})$ est un repère affine pour $\ens{F}$ s'il satisfait une des trois conditions équivalentes:
      \begin{enumerate}[<+(1)->]
        \item $\{A_{0},\ldots,A_{k}\}$ est affinement libre et génératrice pour $\ens{F}$.
        \item $\{A_{0},\ldots,A_{k}\}$ est une famille génératrice minimale pour $\ens{F}$.
        \item $\{A_{0},\ldots,A_{k}\}$ est une famille libre maximale de $\ens{F}$.
      \end{enumerate}
    \end{proposition}
  \end{frame}
% =======================================
\section{Applications affines}
% =======================================
% ~~~~~~~~~~~~~~~~~~~~~~~~~~~~~~~~~~~~~~~
\subsection{Définition}
% ~~~~~~~~~~~~~~~~~~~~~~~~~~~~~~~~~~~~~~~
  % -----------
  \begin{frame}{Définition d'une application affine}
    Soient $\ens{E}$ et $\ens{F}$ deux espaces affines de directions $\ev{E}$ et $\ev{F}$.
    \begin{defprop}
      Une application $\phi:\ens{E}\rightarrow\ens{F}$ est dite \alert{affine} si elle satisfait une des trois conditions équivalentes:
      \begin{enumerate}[<+(1)->]
        \item $\exists(\forall) \Omega \in \ens{E}$, $\phi\in\lin(\ens{E}_{\Omega},\ens{F}_{\phi(\Omega)})$.
        \item $\exists \vv\phi \in \lin(\ev{E},\ev{F})$ telle que $\forall A,B \in \ens{E}$,
          $$
            \vv\phi(\vv{AB})=\vv{\phi(A)\phi(B)} \uncover<+(1)->{\Leftrightarrow \phi(A+\vv{v})=\phi(A)+\vv\phi(\vv{v}).}
          $$
        ($\vv\phi$ est unique et est appelée \myemph{partie linéaire} de $\phi$.)
        \item $\phi$ préserve les barycentres, c.-à.-d. pour $\sum_{i=0}^{k}\mu_{i}=1$
          $$
            \phi(\sum_{i=0}^{k}\mu_{i}A_{i})=\sum_{i=0}^{k}\mu_{i}\phi(A_{i}).
          $$
      \end{enumerate}\pause
      L'ensemble des applications affines est noté $\aff\left(\ens{E},\ens{F}\right)$.
    \end{defprop}
  \end{frame}
% ~~~~~~~~~~~~~~~~~~~~~~~~~~~~~~~~~~~~~~~
\subsection{Exemples}
% ~~~~~~~~~~~~~~~~~~~~~~~~~~~~~~~~~~~~~~~
  % -----------
  \begin{frame}{Exemples d'applications affines}
    \begin{enumerate}[<+(1)->]
      \item Les applications constantes sont affines, de partie vectorielle $0$.
      \item Les applications affines de $\mathbb{R}$ dans $\mathbb{R}$ sont de la forme $x \mapsto ax+b$.
      \item Les applications affines de $\mathbb{R}^{n}$ dans $\mathbb{R}^{m}$ sont de la forme $X \mapsto AX+B$, où $M \in \mathcal{M}_{m,n}$ et $B \in \mathbb{R}^{m}$.
      \item Les applications affines de $\mathbb{C}$ dans $\mathbb{C}$, vu comme $\mathbb{R}$-espace vectoriel, sont de la forme $z \mapsto az+b\overline{z}+c$, où $a,b,c \in \mathbb{C}$.
      \item Les translations $T_{\vv{v}}:M \mapsto M+\vv{v}$ (où $\vv{v} \in \vv{E}$) sont des automorphismes affines de $E$.
      \item Soient $\ev{E}$ et $\ev{F}$ deux espaces vectoriels. Les applications affines de $\aff(\ev{E},\ev{F})$ sont toutes de la forme $\vv{x} \mapsto \vv{\phi}(\vv{x}) + \vv{v}=T_{\vv{v}}\circ\vv{\phi}(\vv{x})$, où $\vv{\phi} \in \lin(\ev{E},\ev{F})$ est linéaire.
    \end{enumerate}
  \end{frame}

% ~~~~~~~~~~~~~~~~~~~~~~~~~~~~~~~~~~~~~~~
\subsection{Propriétés}
% ~~~~~~~~~~~~~~~~~~~~~~~~~~~~~~~~~~~~~~~
  % -----------
  \begin{frame}{Premières propriétés}
    \begin{proposition}
      Soit $\phi\in\aff(\ens{E},\ens{F})$ et $\psi\in\aff(\ens{F},\ens{G})$, alors $\psi\circ\phi\in\aff(\ens{E},\ens{G})$ et a pour partie linéaire $\vv{\psi}\circ\vv{\phi}$.
    \end{proposition}\pause
    \begin{proposition}
      Soit $\phi\in\aff(\ens{E},\ens{F})$, $\ens{A} \subset \ens{E}$ et $\ens{B} \subset \ens{F}$.
      \begin{enumerate}
        \item $\phi(\ens{A})$ est un s.e.a. de $\ens{F}$ de direction $\vv\phi(\ev{A})$.
        \item $\phi^{-1}(\ens{B})$ est vide ou un s.e.a. de $\ens{E}$ de direction $\vv\phi^{-1}(\ev{B})$.
      \end{enumerate}
      Ainsi les images de trois points alignés sont alignées.
    \end{proposition}\pause
    \begin{proposition}
      Pour donner une application affine il suffit de donner :
      \begin{enumerate}[<+(1)->]
        \item la partie linéaire et l'image d'un point,
        \item ou l'image d'un repère.
      \end{enumerate}
    \end{proposition}
  \end{frame}
  % -----------
  \begin{frame}{Les translations (définition)}
    \begin{defprop}
      Une \alert{translation} est  une application affine $T \in \aff(\ens{E})$ qui satisfait une des conditions équivalentes :
      \begin{enumerate}[<+(1)->]
        \item elle est de la forme $T=T_{\vv{v}}:M \mapsto M+\vv{v}$, où $\vv{v} \in \ev{E}$,
        \item sa partie linéaire est $\vv{\phi}=\id \in \lin(\ev{E})$.
      \end{enumerate}
    \end{defprop}
  \end{frame}
  % -----------
  \begin{frame}{Les translations (propriétés)}
    \begin{enumerate}[<+(1)->]
      \item Une translation qui fixe un point est l'identité.
      \item $T_{\vv{u}}\circ T_{\vv{v}} = T_{\vv{u}+\vv{v}}$ : les translations forment un groupe abélien isomorphe à $\ev{E}$.
      \item Les translations de $\mathbb{C}$ sont de la forme $z \mapsto z+c$, pour $c \in \mathbb{C}$.
      \item Soit $\phi \in \aut(\ens{E})$ un automorphisme affine de $\ens{E}$ et $\vv{v} \in \ev{E}$, alors $\phi\circ T_{\vv{v}}\circ\phi^{-1}=T_{\vv{\phi}(\vv{v})}$.
    \end{enumerate}
  \end{frame}
  % -----------
  \begin{frame}{Homothéties affines (définition)}
    \begin{defprop}
      Une \alert{homothétie affine de rapport $\lambda$ et de centre $\Omega$} est une application affine de $H \in\aff(\ens{E})$ qui satisfait une des conditions équivalentes :
      \begin{itemize}[<+(1)->]
        \item $H$ est une homothétie vectorielle de $\ens{E}_{\Omega}$ de rapport $\lambda$;
        \item $H$ fixe $\Omega$ et $\vv{H}=\lambda\id \in \lin(\ev{E})$;
        \item elle est de la forme $H=H_{\Omega,\lambda}:M \mapsto \lambda M + (1- \lambda)\Omega$.
      \end{itemize}
    \end{defprop}
  \end{frame}
  % -----------
  \begin{frame}{Homothéties affines (propriétés)}
    \begin{enumerate}[<+(1)->]
      \item Une homothétie qui fixe deux points est l'identité.
      \item Si $\vv{H}=\lambda\id$ avec $\lambda\neq1$, alors $H$ est une homothétie affine.
      \item La composée de deux homothéties, l'une de rapport $\lambda$ et l'autre de rapport $\mu$, est :
      \begin{itemize}[<+(1)->]
        \item Une homothétie de rapport $\lambda\mu$, si $\lambda\mu \neq 1$.
        \item Une translation, si $\lambda\mu=1$.
      \end{itemize}
      \item Les homothéties $h_{\omega,\lambda}$ du $\mathbb{R}$-espace vectoriel $\mathbb{C}$ sont de la forme $z \mapsto \lambda z + (1-\lambda)w$, pour $\lambda \in \mathbb{R}, \omega \in \mathbb{C}$.
      \item Soit $\phi \in \aut(\ens{E})$ un automorphisme affine de $\ens{E}$ et $h_{\Omega,\lambda}$ une homothétie de centre $\Omega$ et de rapport $\lambda$, alors $\phi\circ h_{\Omega,\lambda}\circ\phi^{-1}=h_{\phi(\Omega),\lambda}$.
    \end{enumerate}
  \end{frame}
  % -----------
  \begin{frame}{Les points fixes}
    \begin{proposition}
      Soit $\dim \ens{E} < \infty$, alors $\phi\in\aff(\ens{E})$ possède un unique point fixe ssi $\vv{\phi}$ possède un unique point fixe (forcément $0\in\ev{E}$)\pause, autrement dit, ssi $1 \notin Sp(\vv{\phi})$.
    \end{proposition}\pause
    \begin{proposition}
      Soit $\vv*{\ens{E}}{1} \neq 0$ l'ensemble de points fixes de $\vv{\phi}$, alors
      \begin{enumerate}[<+(1)->]
        \item si $\phi$ possède un point fixe $\Omega$, l'ensemble de points fixes de $\phi$ est $\Omega+\vv*{\ens{E}}{1}$;
        \item si $\phi$ n'a pas de points fixes\pause, et
          $$
            \ker(\vv{\phi}-\id) \oplus \im(\vv{\phi}-\id) = \ev{E}
          $$\pause
        alors il existe un unique $\vv{v}\in\vv*{\ens{E}}{1}$ tel que $T_{\vv{v}}\circ\phi=\phi\circ T_{\vv{v}}$\pause\\
        et $T_{\vv{v}}\circ\phi$ possède \uncover<+(1)->{(au moins)} un point fixe.
      \end{enumerate}
    \end{proposition}
  \end{frame}
% ~~~~~~~~~~~~~~~~~~~~~~~~~~~~~~~~~~~~~~~
\subsection{GA}
% ~~~~~~~~~~~~~~~~~~~~~~~~~~~~~~~~~~~~~~~
  % -----------
  \begin{frame}{Le groupe affine}
    \begin{proposition}
      Soit $\phi\in\aff(\ens{E})$, alors $\phi$ est une bijection ssi $\vv\phi$ l'est, et dans ce cas $\phi^{-1}$ est une application affine avec partie linéaire $\big(\vv\phi\big)^{-1}$.
    \end{proposition}\pause
    \begin{proposition}
      Les bijections affines de $\ens{E}$ dans lui-même forment un groupe, le groupe affine $GA(\ens{E})$.\pause\ Et l'application $\phi\mapsto\vv\phi$ est un morphisme surjectif de groupes $GA(\ens{E})\twoheadrightarrow GL(\ev{E})$\pause, de noyau le sous-groupe abélien des translations de $\ens{E}$.
    \end{proposition}
  \end{frame}
% =======================================
\section{Convexes}
% =======================================
% ~~~~~~~~~~~~~~~~~~~~~~~~~~~~~~~~~~~~~~~
\subsection{Définition}
% ~~~~~~~~~~~~~~~~~~~~~~~~~~~~~~~~~~~~~~~
% ---------------
\begin{frame}{Définition d'un convexe}
  \begin{definition}
    Soient $A$ et $B$ deux points d'un espace affine. On note $\convhull{AB}=\left\{\lambda A + (1-\lambda)B\,|\, \lambda \in [0,1]\right\}$ l'ensemble des barycentres à poids positifs, appelé \myemph{le segment} $\convhull{AB}$.
  \end{definition}\pause
  \begin{definition}
    On dit que $\ens{C}$ est un ensemble \myemph{convexe}, si pour tous deux points $A,B \in \ens{C}$ le segment $\convhull{AB}$ est entièrement contenu dans $\ens{C}$.
  \end{definition}\pause
  \begin{proposition}
    Un ensemble $\ens{C}$ est convexe ssi tout barycentre de points de $\ens{C}$ à poids \textbf{positifs} est dans $\ens{C}$.
  \end{proposition}
\end{frame}

% ~~~~~~~~~~~~~~~~~~~~~~~~~~~~~~~~~~~~~~~
\subsection{Propriétés}
% ~~~~~~~~~~~~~~~~~~~~~~~~~~~~~~~~~~~~~~~
% ---------------
\begin{frame}{Propriétés}
  \begin{enumerate}[<+(1)->]
    \item L'intersection d'ensembles convexes est convexe.
    \item L'ensemble vide et les ensembles à un point sont convexes.
    \item Un sous-espace affine est convexe.
    \item Les demi-espaces (ouverts, fermés) sont convexes.
    \item L'image d'un convexe par une application affine est convexe.
    \item L'image réciproque d'un convexe par une application affine est convexe.
    \item Une fonction réelle est convexe ssi la partie au-dessus du graphe est convexe.
  \end{enumerate}
\end{frame}

% ~~~~~~~~~~~~~~~~~~~~~~~~~~~~~~~~~~~~~~~
\subsection{Enveloppe convexe}
% ~~~~~~~~~~~~~~~~~~~~~~~~~~~~~~~~~~~~~~~
% ---------------
\begin{frame}{Enveloppe convexe}
  \begin{defprop}
    Soit $\ens{A}$ une partie d'un espace affine. L'enveloppe convexe, noté $\convhull{\ens{A}}$, est :
    \begin{enumerate}[<+(1)->]
      \item Le plus petit convexe contenant $\ens{A}$.
      \item L'intersection de tous les convexes contenant $\ens{A}$.
      \item L'ensemble de barycentres de points de $\ens{A}$ de poids positifs.
    \end{enumerate}
  \end{defprop}\pause
    Ainsi par exemple le segment $\convhull{AB}$ est l'enveloppe convexe de $\left\{A,B\right\}$.
\end{frame}

\end{document}
