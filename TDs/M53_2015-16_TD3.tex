\documentclass[a4paper,12pt,reqno]{amsart}
\usepackage{graphicx}
\usepackage{macros_M53}

\begin{document}

\hautdepage{TD3: Nombres complexes et géométrie}

% ==================================
\section{Géométrie élémentaire}
% ==================================


%-----------------------------------
\begin{exo} (Conditions géométriques)
  \begin{enumerate}
    \item Étant donnés deux points $A$ et $B$ du plan complexe d'affixes $a$ et $b$, donner une condition sur $a\overline{b}$ pour que la droite $AB$ passe par $O$ d'affixe $0$. Préciser quand $O$ est entre $A$ et $B$, et quand il ne l'est pas.

    \item Donner une condition nécessaire et suffisante sur les affixes $a$, $b$, $c$ de trois points $A$, $B$, $C$ du plan complexe pour que le triangle $ABC$ soit équilatéral.
  \end{enumerate}
\end{exo}


%-----------------------------------
\begin{exo} (Bac C -- 1991)

  Le plan complexe $P$ est identifié avec $\mathbb{C}$. Soit $f$ l'application de $P$ vers $P$ définie par
    $$
      f(z)=\frac{3+i\sqrt{3}}{4}z+\frac{1-i\sqrt{3}}{2}
    $$
  \begin{enumerate}
    \item Déterminer la nature géométrique et les éléments caractéristiques de $f$.
    \item Montrer que le triangle $\Omega M\! f(M)$ est rectangle, en supposant que $\Omega$ est un point fixe de $f$.
  \end{enumerate}
\end{exo}


%-----------------------------------
\begin{exo}

  \image{l}{7cm}{-17mm}{0mm}{M53_2015-16_TD3_exo03.tikz}

  À l'extérieur d'un triangle $ABC$, on construit trois carrés de bases les côtés et de centres $P$, $Q$ et $R$. Montrer que les segments $AP$ et $QR$ (resp. $BQ$ et $RP$, $CR$ et $PQ$) sont orthogonaux et de même longueur. En déduire que les droites $AP$, $BQ$ et $CR$ sont concourantes.
\end{exo}

%-----------------------------------
\begin{exo}

  \image{r}{7cm}{-17mm}{0mm}{M53_2015-16_TD3_exo04.tikz}

  On construit à l'extérieur d'un parallélogramme $ABCD$ quatre carrés de bases les côtés et de centres $M$, $N$, $P$ et $Q$. Montrer que $MNPQ$ est un carré.
\end{exo}



% ==================================
\section{Équations}
% ==================================


%-----------------------------------
\begin{exo} (Polygone des milieux)

  Reprendre l'exercice 9.c) de la feuille de td n°1 en traduisant le problème en un système de $n$ équations linéaires à $n$ inconnues dans $\mathbb{C}$. Discuter le rang de ce système selon la parité de $n$.
\end{exo}


%-----------------------------------
\begin{exo} (Équation d'une droite)

  On se place dans $\mathbb{C}$ que l'on considère comme un plan affine réel. On se propose de retrouver la formule d'une droite affine
    \begin{equation}\tag{\dag}\label{eq:droite}
      \overline{\beta}z+\beta\overline{z}+\gamma=0
      \text{,\qquad avec } \beta \in \mathbb{C} \text{ et } \gamma \in \mathbb{R}.
    \end{equation}

  en donnant une équation complexe de la droite $\ens{D}$ passant par deux points $A$ et $B$.
  \begin{enumerate}
    \item Montrer que si $M$ est aligné avec $A$ et $B$ alors $(z-a)(\overline{z}-\overline{b})$ est réel, où $z$ est l'affixe de $M$, $a$ celle de $A$ et $b$ celle de $B$.
    \item Montrer que $(z-a)(\overline{z}-\overline{b})$ est réel est équivalent à $z$ satisfait
      $$
        (z-a)(\overline{z}-\overline{b})=(\overline{z}-\overline{a})(z-b),
      $$
      puis en posant $\beta=i(a-b)$ en déduire que c'est équivalent à $z$ satisfait \eqref{eq:droite} pour un certain $\gamma \in \mathbb{R}$ à preciser.
  \end{enumerate}
\end{exo}

%-----------------------------------
\begin{exo} (Équation d'un cercle)

  Montrer que tout cercle du plan complexe est défini par une équation de la forme
    $$
      z\overline{z}-a\overline{z}-\overline{a}z+c=0,
    $$
  où $a$ est un nombre complexe et $c$ un réel vérifiant $c \leq |a|^{2}$. Montrer que réciproquement toute équation de ce type est celle d'un cercle.
\end{exo}


%-----------------------------------
\begin{exo} (Encore des équations)
  \begin{enumerate}
    \item Discuter selon les valeurs de $\alpha,\gamma \in \mathbb{R}$ et $\beta \in \mathbb{C}$ quel est le sous-ensemble de $\mathbb{C}$ défini par l'équation
      $$
        \alpha.z\overline{z}+\beta\overline{z}+\overline{\beta}z+\gamma=0.
      $$
    \item Soit $\lambda$ un nombre réel positif, décrire géométriquement l'ensemble
      $$
        E_{\lambda}=\lbrace z\in\mathbb{C}:\vert z-a \vert =\lambda \vert z-b \vert \rbrace .
      $$
    \item Déterminer l'ensemble des nombres complexes tels que $\vert z-1 \vert=\vert z-i z\vert=\vert z-i \vert$.
  \end{enumerate}
\end{exo}



% ==================================
\section{Géométrie projective sans le dire}
% ==================================


%-----------------------------------
\begin{exo} (points cocycliques et le théorème de Ptolémée)

  On se donne $4$ points distincts $A$, $B$, $C$ et $D$ d'affixes $a$, $b$, $c$ et $d$ respectivement.
  \begin{enumerate}
    \item Montrer que l'on a toujours
      $$
        |AC|.|BD| \leq |AB|.|CD|+|AD|.|BC|
      $$
      avec égalité si et seulement s'il existe un réel $\lambda > 0$ tel que $(b-a)(d-c)=\lambda (d-a)(c-b)$.
    \item Déduire de la question précédente que le cas d'égalité est équivalent à
      $$
        \arg(\frac{b-a}{d-a})-\arg(\frac{b-c}{d-c}) = \pm\pi.
      $$
    \item Montrer le théorème de Ptolémée :

    \textit{Un quadrilatère convexe est inscrit dans un cercle si et seulement si le produit des longueurs de ses diagonales est égal à la somme des produits des longueurs des côtés opposés.}
  \end{enumerate}
 \end{exo}


%-----------------------------------
\begin{exo} (Birapports)

  Soit $4$ points du plan distincts deux à deux avec affixes $a$, $b$, $c$, $d$. On définit leur birapport comme étant la quantité:
  $$
    [a,b,c,d]=\frac{a-c}{b-c}\cdot\frac{b-d}{a-d} \in \mathbb{C}.
  $$
  \begin{enumerate}
    \item Montrer que si $4$ points sont alignés alors leur birapport est un nombre réel. Dans quel cas ce birapport est positif ?
    \item Montrer que si $4$ points sont cocycliques alors leur birapport est un nombre réel. Dans quel cas ce birapport est positif ?
    \item Montrer que le birapport de $4$ points est réel si et seulement si les $4$ points sont alignés ou cocycliques.
    \item Prouver que si $4$ points d'affixes non nuls sont alignés alors leurs images par $1/\overline{z}$ sont alignées ou cocycliques.
    \item Montrer que les homothéties non nulles, les rotations et les translations préservent les birapports, et que les symétries les transforment en leurs conjugués.
    \item Montrer que les points d'affixes $-2+6i$, $1+7i$, $4+6i$, $6+2i$ sont cocycliques.
  \end{enumerate}
\end{exo}


%-----------------------------------
\begin{exo} (Inversions)

  Une inversion $i(\Omega,r)$ de centre $\Omega$ et de rapport $r \in \mathbb{R}^{*}_{+}$ est une application du plan $P$ privé de $\Omega$ dans $P$ qui à tout point $M$ associe le point $M'$ tel que
  \begin{itemize}
    \item $M'$ soit sur la droite $\Omega M$,
    \item $\overline{\Omega M}.\overline{\Omega M'}=r^2$.
  \end{itemize}
  \begin{enumerate}
    \item Déterminer l'image de $i(\Omega,r)$.
    \item Montrer que l'inversion est une involution.
    \item En identifiant le plan $P$ avec $\mathbb{C}$, donner l'expression analytique de l'inversion $i(0,1)$, puis, plus généralement, de $i(\omega,r)$ avec $\omega \in \mathbb{C}$ et $r \in \mathbb{R}^{*}_{+}$.
    \item  Montrer que chaque point du cercle de centre $\Omega$ et de rayon $r$ est invariant par l'inversion.
    \item Montrer qu'une inversion transforme un birapport en son conjugué.
    \item Montrer que l'inversion préserve les droites qui passent par $\Omega$.
    \item Montrer qu'elle envoie une droite qui ne passe par $\Omega$ sur un cercle qui passe par $\Omega$.
    \item Montrer aussi qu'elle envoie tout cercle qui ne passe pas par $\Omega$ sur un cercle qui ne passe pas par $\Omega$.
  \end{enumerate}
\end{exo}


%-----------------------------------
\begin{exo} (Homographies)

  Une homographie est une application de $\mathbb{C}$ dans $\mathbb{C}$ de la forme
  $$
    z\mapsto \frac{az+b}{cz+d}
  $$
  avec $a,b,c,d$ complexes, tels que $ad-bc\neq 0$.
  \begin{enumerate}
    \item Montrer que toute homographie est la composée de translations, d'homothéties et au plus d'une inversion et d'une réflexion.
    \item Montrer que toute homographie envoie droites ou cercles sur droites ou cercles.
    \item Montrer que les homographies forment un groupe.
  \end{enumerate}
\end{exo}


%-----------------------------------
\begin{exo}

  \image{r}{7cm}{-17mm}{0mm}{M53_2015-16_TD3_exo13.tikz}

  Deux cercles $\mathcal{C}$ et $\mathcal{C}'$ sont tangents intérieurement en $O$. On considère une famille de cercles deux à deux tangents entre eux et tangents à $\mathcal{C}$ et $\mathcal{C}'$. Montrer que les points de contact de ces cercles entre eux sont situés sur un même cercle tangent en $O$ aux deux cercles $\mathcal{C}$ et $\mathcal{C}'$.
\end{exo}

\end{document}
