\documentclass[a4paper,12pt,reqno]{amsart}
\usepackage{macros_M53}

\begin{document}

\hautdepage{TD4: Coniques et quadriques}

% ==================================
\section{Ellipses}
% ==================================

%-----------------------------------
\begin{exo} (Paramétrisation d'une ellipse)

  \image{r}{4cm}{-13mm}{-7mm}{M53_2015-16_TD4_ellipse_clous_corde.tikz}

  Soit une ellipse $E = \{ M \in \mathbb{R}^{2} \,|\, \d(F_{1},M)+\d(M,F_{2})=2a\}$ définie par ses deux foyers $F_{1}$ et $F_{2}$ à distance $\d(F_{1},F_{2})=2c$, et une constante réelle $a > c$.

  Déterminer une courbe paramétrée $f:\mathbb{R}\to\mathbb{R}^{2}$ dont l'image est $E$. Donner l'expression explicite de $f$ dans le cas où $F_{1}=(-2,-1)$, $F_{2}=(2,2)$ et $a=3$.
\end{exo}

%-----------------------------------
\begin{exo} (Grand diamètre d'une ellipse)

  Soit dans un plan affine euclidien une ellipse $E$ qui n'est pas un cercle. Montrer qu'il existe un unique couple (non ordonné) de points $\{M_{1},M_{2}\}$ tel que
  \[
    \d(M_{1},M_{2})=\max_{A,B \in E}\d(A,B).
  \]
  En déduire qu'il n'y a que deux symétries orthogonales qui préservent cette ellipse.
\end{exo}


%-------------------------------------
\begin{exo} (Projection d'un cercle)

  \begin{enumerate}
    \item  Soit $P$ un plan d'équation cartésienne $ax+by+cz+d=0$, donner une condition sur $a,b,c,d$ pour que $P$ ne soit pas orthogonal au plan $z=0$.
    \item Soit $A\in P$, donner des équations du cercle $C\subset P$ de centre $A$ et de rayon $R$.
    \item On suppose que $P$ n'est pas orthogonal à $z=0$, montrer que le projeté orthogonal du cercle $C$ sur le plan $z=0$ est une ellipse.  À quelle condition est-ce un cercle?
  \end{enumerate}

\end{exo}


%-------------------------------------
\begin{exo} (Théorème de Hire)

  \image{r}{4cm}{-17mm}{0mm}{M53_2015-16_TD4_hire.tikz}

  Soient un disque de rayon $R$ qui roule à l'intérieur d'un cercle de rayon $2R$ et $M$ un point fixe du premier disque. Montrer que $M$ décrit une ellipse et réciproquement que toute ellipse peut être obtenue de cette façon.

\end{exo}


% ==================================
\section{Paraboles et hyperboles}
% ==================================

%---------------------------------------
\begin{exo} (Miroir parabolique)

  \image{r}{5cm}{-17mm}{0mm}{M53_2015-16_TD4_parabole_tangente.tikz}

  Soient une droite $\mathcal{D}$ et un point $F$ en dehors de cette droite. On rappelle qu'une parabole de directrice $\mathcal{D}$ et de foyer $F$ est l'ensemble des points $M$ tels qu'on ait l'égalité des distances
  $$
    d(M,F)= d(M,\mathcal{D}).
  $$

  \image{r}{3.5cm}{35mm}{-70mm}{M53_2015-16_TD4_parabole_miroir.tikz}

  On se place dans le repère affine orthonormé $(O,\vv{u},\vv{v})$, où $\vv{u}$ et $\vv{v}$ sont des vecteurs directeurs respectivement de l'axe focal et de la directrice. Soient $(a,0)$ les coordonnées de $F$ dans ce repère.

  \begin{enumerate}

    \item Déterminer l'équation de la parabole dans ce repère.

    \item Donner une courbe paramétrée $f:\mathbb{R}\to\mathbb{R}^{2}$ dont l'image est cette parabole.

    \item Soient $M$ un point de la parabole et $P$ sa projection sur la directrice.\\
    Montrer que la tangente en $M$ est orthogonale à $FP$.

    \item En déduire que la tangente en $M$ est la bissectrice de l'angle $\widehat{FMP}$.

    \item Montrer qu'étant donné un miroir concave parabolique, les rayons\\
    parallèles à l'axe focal passent par le foyer.
  \end{enumerate}

\end{exo}

%-----------------------------------
\begin{exo} (Paramétrisation d'une hyperbole)

  % \image{r}{4cm}{-11mm}{0mm}{M53_2015-16_TD4_ellipse_clous_corde.tikz}

  Soit une hyperbole $H = \{ M \in \mathbb{R}^{2} \,|\ \big|\d(F_{1},M)-\d(M,F_{2})\big|=2a\}$ définie par ses deux foyers $F_{1}$ et $F_{2}$ à distance $\d(F_{1},F_{2})=2c$, et une constante réelle $a \in ]0,c[$.

  Déterminer une courbe paramétrée $f:\mathbb{R}\to\mathbb{R}^{2}$ dont l'image est la branche $\{ M \in \mathbb{R}^{2} \,|\, \d(F_{1},M)-\d(M,F_{2})=2a\}$ de $H$.
\end{exo}


%---------------------------------------
\begin{exo} ( Parallélogrammes à surface constante)

  \image{r}{5cm}{-7mm}{0mm}{M53_2015-16_TD4_hyperbole_surface.tikz}

  Soient deux droites qui se coupent en un point $O$. On considère l'ensemble des points $M$ tels que le parallélogramme ayant $O$ et $M$ comme sommets opposés et deux côtés sur les droites données a une surface fixé $S>0$.\\
  Montrer que cet ensemble est formé par deux hyperboles ayant les deux droites comme asymptotes. \\
  Et réciproquement, étant donné une hyperbole, montrer que les parallélogrammes formés par les asymptotes, le centre et les points $M$ de l'hyperbole, ont une surface constante (ne dépendant pas du choix de $M$).
\end{exo}

%---------------------------------------
\begin{exo} (Sécantes et tangentes à une hyperbole)

\image{r}{5cm}{-17mm}{0mm}{M53_2015-16_TD4_hyperbole_secantes.tikz}

On rappelle que l'équation de l'hyperbole dans le repère affine orthonormé $(O,\vv{u},\vv{v})$, où $\vv{u}$ et $\vv{v}$ sont des vecteurs directeurs respectivement de l'axe focal et de la directrice, s'écrit
$$
  \frac{x^{2}}{a^{2}}-\frac{y^{2}}{b^{2}}=1
$$

\begin{enumerate}
  \item Montrer que toute droite parallèle à l'une des asymptotes de l'hyperbole (et distincte de cette asymptote) coupe l'hyperbole en exactement un point.

  \item Montrer que si une droite $\mathcal{D}$ coupe l'hyperbole en deux points $M$ et $N$ et les asymptotes en deux points $P$ et $Q$, les segments MN et PQ ont le même milieu (on écrira une équation du second degré dont les abscisses des points $M$ et $N$ (resp. $P$ et $Q$) sont solutions et on évaluera la demi-somme de ces solutions).

  \item En déduire une construction de l'hyperbole point par point connaissant les asymptotes et un point.

  \item Montrer que les milieux des cordes $MN$ de l'hyperbole parallèles à une direction donnée appartiennent tous à une même droite passant par le centre $O$ de l'hyperbole.

  \item Soit $\tau$ une tangente à l'hyperbole en $T$, $R$ et $S$ les points d'intersection de $\tau$ avec les asymptotes. Montrer que $| TR | = | TS |$.

  \item Montrer que l'aire du triangle $ORS$ ne dépend pas de $T$.
\end{enumerate}

\end{exo}

% ==================================
\section{Coniques et quadriques}
% ==================================


%---------------------------------------
\begin{exo} (Équations et natures des coniques)

  Déterminer la nature des courbes suivantes et, quand il existe, leur centre.
  \begin{examplescol}{2}
    \item $(x-y+1)^2+(x+y-1)^2=0$,
    \item $x(x-1)+(y-2)(y-3)=0$,
    \item $4x^{2}+y^{2}+4xy+2x+3y-1$.
    \item $(x+y+1)(x-y+3)=3$.
    \item $2x^{2}+4xy+y^{2}+x+y=0$.
  \end{examplescol}

\end{exo}


%----------------------------------------
\begin{exo} (Quadriques)

  Dessiner les surfaces:
  \begin{examplescol}{2}
    \item $x^2-2y^2+2z^2=1$
    \item $xy=1$
    \item $x^2+y^2+4z^2=6$
    \item $z-4xy=0$
    \item $z^2-4xy=0$
  \end{examplescol}

\end{exo}


%----------------------------------------
\begin{exo} (Hyperboloïde à une nappe)

  On considère l'hyperboloïde standard à une nappe $H$ défini par l'équation $x^{2}+y^{2}-z^{2}=1$.
  \begin{enumerate}
    \item Montrer que l'image de $t\mapsto[\cos(u)-t\sin(u),\sin(u)+t\cos(u),t]\in \mathbb{R}^{3}$ est une droite contenue dans $H$.
    \item Montrer que $H$ est une réunion de droites deux à deux disjointes.
    \item Déterminer une autre famille de droites dont la réunion est $H$.
    \item Est-ce que tout hyperboloïde à une nappe est la réunion de droites ?
    \item Soit $\ens{D}$ une droite non coplanaire avec $Oz$. Soit $R_{\theta}$ la rotation de $\theta$ autour de l'axe $Oz$. Montrer que $\bigcup_{\theta}R_{\theta}(\ens{D})$ est un hyperboloïde à une nappe.
  \end{enumerate}

\end{exo}

\end{document}
