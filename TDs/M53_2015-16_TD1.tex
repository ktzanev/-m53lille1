\documentclass[a4paper,12pt,reqno]{amsart}
\usepackage{macros_M53}

\begin{document}

\hautdepage{TD1: Espaces affines, notions de base}

% ==================================
\section{Espaces et sous-espaces affines}
% ==================================

%-----------------------------------
\begin{exo} (Droites)

  \begin{enumerate}
    \item\label{q:reel} On considère l'espace affine $\mathbb{R}^{2}$. Soit $\mathcal{D}$ la droite $\big\{(x,y) \;\big|\; x+2y=1\big\}$.
    \begin{enumerate}
      \item Déterminer toutes les équations cartésiennes qui définissent la même droite $\mathcal{D}$.
      \item Déterminer la direction de $\mathcal{D}$.
      \item Déterminer toutes les équations cartésiennes qui définissent une droite parallèle à $\mathcal{D}$.
      \item Déterminer les équations cartésiennes qui définissent la droite parallèle à $\mathcal{D}$ qui passe par $A=(2,1)$.
      \item Déterminer une équation de la droite qui passe par les points $C=(2,3)$ et $D=(4,-3)$. Est-elle parallèle à $\mathcal{D}$?
    \end{enumerate}
    \item On considère le plan complexe $\mathbb{C}$ comme un espace affine réel.
    \begin{enumerate}
      \item Montrer que toute droite réelle de $\mathbb{C}$ est définie par une équation complexe de la forme
        \[
          \big\{z \in \mathbb{C} \;\big|\; \overline{\beta}z+\beta\overline{z}+\gamma=0\big\}
        \]
        où $\beta \in \mathbb{C}^{*}$ et $\gamma \in \mathbb{R}$.
      \item Déterminer une équation complexe de la droite $\mathcal{D}$ de la question (\ref{q:reel}).
      \item Donner une condition sur les coefficients des équations complexes pour que deux droites soient parallèles.
    \end{enumerate}
    \item On considère l'espace affine réel $\mathbb{R}^{3}$.
    \begin{enumerate}
      \item Déterminer des équations qui définissent la droite $\mathcal{T}$ qui passe par les deux points $M=(1,0,1)$ et $N=(-1,1,1)$.
      \item Déterminer la direction de $\mathcal{T}$
      \item Déterminer des équations de la droite parallèle à $\mathcal{T}$ qui passe par le point $P=(-1,2,1)$.
      \item Trouver une droite qui n'est ni parallèle à $\mathcal{T}$, ni sécante avec $\mathcal{T}$.
    \end{enumerate}
  \end{enumerate}
\end{exo}

%-----------------------------------
\begin{exo} (Barycentres)

On se place dans $\mathbb{R}^{2}$.

\begin{enumerate}
  \item Calculer l'isobarycentre des points $A=(1,-2)$, $B=(0,-2)$, $C=(2,-4)$. Puis le barycentre des mêmes points affectés des poids suivants: $(A,2)$, $(B,4)$, $(C,-1)$.

  \item Montrer que les médianes d'un triangle sont concourantes.

  \item Discuter la position d'un point $M$ par rapport au triangle non dégénéré $ABC$, en fonction des signes de ses coordonnées barycentriques $[\alpha, \beta, \gamma]$.

\end{enumerate}

\end{exo}

%-----------------------------------
\begin{exo} (Exemples d'espaces affines)
  \begin{enumerate}
    \item Montrer que pour tout $n \in \mathbb{N}^\ast$ l'ensemble $\ens{E}_{n}$ des polynômes unitaires de degré $n$ à coefficients réels est un espace affine. Quelle est sa dimension? En donner un repère affine.
    \item Soient $d$ un entier non nul, et $a_1,\ldots, a_d$ et $b$ des réels. Soit $\ens{E}$ l'ensemble des suites réelles $u=\big(u_n\big)_{n \in \mathbb{N}}$ vérifiant
      \[
        u_{n+d} = a_{1} u_{n+d-1} + \cdots + a_{d-1} u_{n+1} + a_d u_n + b \quad \text{pour tout }n \in \mathbb{N}.
      \]
      \begin{enumerate}
        \item Montrer que $\ens{E}$ est un espace affine.
        \item Quelle est sa dimension?
        \item\hard Décrire les éléments de $\ens{E}$ lorsque $d=1$ (distinguer les cas $a_1 \neq 1$ et $a_1=1$).
        \item\hard On suppose maintenant $d=2$. Donner une base de la direction $\vv{\ens{E}}$ de $\ens{E}$. Donner un repère affine de $\ens{E}$ dans le cas où $a_1+a_2 \neq 1$.
      \end{enumerate}
    \end{enumerate}
\end{exo}

%-----------------------------------
\begin{exo} (EDO et espaces affines)

  Dans ce qui suit on s'intéresse aux structures portées par les espaces de solutions des équations différentielles dans $C^{\infty}(\mathbb{R})$:\vspace{-4mm}
    \begin{align*}
      y' - y &= 4 \qquad (G)\\
      y' - y &= 0 \qquad (H)
    \end{align*}

  \begin{enumerate}
    \item Montrer que l'espace des solutions $\mathcal{S}_{H}$ de $(H)$ est un $\mathbb{R}$-espace vectoriel dont on déterminera la dimension.

    \item L'espace des solutions $\mathcal{S}_{G}$ de $(G)$ est-il un sous-espace vectoriel de $C^{\infty}(\mathbb{R})$?

    \item Déterminer une solution particulière de $(G)$ que l'on notera $C$.

    \item Montrer que toute solution de $(G)$ s'écrit de manière unique sous la forme $f+C$ où $f$ est une solution de $(H)$.

    \item\hard En déduire que les solutions $\mathcal{S}_{G}$ de $(G)$ sont en bijection avec les solutions $\mathcal{S}_{H}$ de $(H)$. Cette bijection est-elle canonique?

  \end{enumerate}
\end{exo}

%-----------------------------------
\begin{exo} (Changement de repères)

  L'espace $\mathbb{R}^{3}$ est muni de son repère cartésien canonique $\mathcal{R}=(O,I,J,K)$. On considère le nouveau repère $\mathcal{R}'=\left( O'=(0, 0, 1), I'=(0, 0, 0), J'=(1, 0, 2), K'=(0, 1, 1) \right)$.
  \begin{enumerate}
    \item Soit un point de coordonnées $X=(x, y, z)$ dans le repère $\mathcal{R}$. Donner ses coordonnées cartésiennes $X'=(x', y', z')$ dans le repère $\mathcal{R}'$ en fonction de $X=(x, y, z)$ .

    \item \label{chrep} Donner la formule de changement de repère de $\mathcal{R}$ à $\mathcal{R'}$ sous la forme:
      \[
        \left(\begin{array}{c} x' \\ y' \\ z'\end{array}\right) =
        M \left(\begin{array}{c} x \\ y \\ z \end{array}\right) +
          \left(\begin{array}{c} u \\ v \\ w \end{array}\right),
      \]
    où $M$ est une matrice $3\times3$. Puis écrire sous la même forme la formule de changement de repère inverse, de $\mathcal{R'}$ à $\mathcal{R}$.

    \item On se donne $A=(0, 0, 3)$, $B=(1, 0, 4)$, $C=(1, 1, 1)$, trois points dont les coordonnées cartésiennes sont exprimées dans $\mathcal{R}$. Donner une équation du plan $\left<A,B,C\right>$ dans le repère $\mathcal{R}$, puis dans le repère $\mathcal{R}'$.

    \item \label{compchrep} De la question \ref{chrep}) on déduit qu'un changement de repères $P_{\mathcal{R}'}^{\mathcal{R}}$ s'écrit sous la forme $X'=P_{\mathcal{R}'}^{\mathcal{R}}(X)=M_{\mathcal{R}'}^{\mathcal{R}}X+V_{\mathcal{R}'}^{\mathcal{R}}$. Soit un troisième repère $\mathcal{R}''$, quelle formule relie les couples $(M_{\mathcal{R}'}^{\mathcal{R}},V_{\mathcal{R}'}^{\mathcal{R}})$, $(M_{\mathcal{R}''}^{\mathcal{R}'},V_{\mathcal{R}''}^{\mathcal{R}'})$ et $(M_{\mathcal{R}''}^{\mathcal{R}},V_{\mathcal{R}''}^{\mathcal{R}})$ ?

    \item\hard On note $GA_{3}(\mathbb{R})$ l'ensemble des couples $(M,V)$ où $M\in GL_{3}(\mathbb{R})$ et $V\in\mathbb{R}^{3}$. On munit $GA_{3}(\mathbb{R})$ d'une opération binaire:\vspace{-3mm}
      \[
        (M',V')\circ(M,V)=(M'M,M'V+V')
      \]
    Montrer que cette opération munit $GA_{3}(\mathbb{R})$ d'une loi de groupe.

    \item\hard Montrer que $GA_{3}(\mathbb{R})$ est en bijection avec l'ensemble des repères de $\mathbb{R}^{3}$. \\
    Cette bijection est-elle canonique?
  \end{enumerate}
\end{exo}

%-----------------------------------
\begin{exo} (Exemples de sous-espaces affines)
  \begin{enumerate}

  \item Soit $X$ un espace topologique, on considère l'espace $C(X,\mathbb{R})$ des fonctions réelles continues sur $X$. Soit $a \in X$ montrer que l'ensemble
    \[
      \mathcal{F}_{a,1}=\{ f \in C(X,\mathbb{R}) : f(a)=1 \}
    \]
  est un sous-espace affine de $C(X,\mathbb{R})$.

  \item Montrer que l'ensemble des matrices
    \[
      \left\{
        \left(
          \begin{array}{cc}
            1-a & b-2a\\
            a+b & 3
          \end{array}
        \right) \in M_{2}(\mathbb{R})
        : a,b \in \mathbb{R}
      \right\}
    \]
  est un sous-espace affine de $M_{2}(\mathbb{R})$. En donner un repère affine.

  \item Montrer que le cercle $\left\{ (x,y) : x^{2}+y^{2} = 1 \right\} $ n'est
  pas un sous-espace affine de $\mathbb{R}^{2}$.

  \item Montrer que $\mathcal{H} = \{ f \in \mathbb{R}^{\mathbb{R}} : f(x+1)=f(x)+1\}$
  est un sous-espace affine de $\mathbb{R}^{\mathbb{R}}$ . Déterminer
  un point de $\mathcal{H}$ et sa direction.
  \end{enumerate}
\end{exo}

%-----------------------------------
\begin{exo} \hard (Des triangles sur un corps fini)

  On se place dans l'espace affine $\mathcal{E}=(\mathbb{F}_{3})^{2}$ sur $\mathbb{F}_{3}=\mathbb{Z}/3\mathbb{Z}$.

  \begin{enumerate}
    \item Combien contient-il de points et de droites? Faire des dessins !

    \item Etant donnés deux points $M$ et $N$ de $\mathcal{E}$, le milieu de $(M,N)$ est le point $\overline{\frac{1}{2}}M+\overline{\frac{1}{2}}N$. On considère les points $A=(\overline{0}, \overline{0})$, $B=(\overline{2}, \overline{0})$, $C=(\overline{0}, \overline{2})$. Déterminer les médianes du triangle $ABC$. Sont-elles concourantes?
  \end{enumerate}
\end{exo}


% ==================================
\section{Applications affines}
% ==================================

%-----------------------------------
\begin{exo} (Projections et symétries)

  On se place dans l'espace affine (euclidien) $\mathbb{R}^{3}$. Le but de cet exercice est de donner des expressions analytiques pour des projections et des symétries. On considère le plan $\mathcal{P}$ d'équation $\{x+y+z=1\}$ et la droite $\mathcal{D}$ d'équations $\{z=4,\ y=0\}$.

  \begin{enumerate}

    \item Donner l'expression analytique de la projection $p$ sur le plan $\mathcal{P}$ suivant la direction $\mathcal{D}$.

    \item Donner l'expression analytique de la symétrie $s$ par rapport à $\mathcal{P}$ suivant la direction $\mathcal{D}$.

    \item Donner l'expression analytique de la projection orthogonale $\pi$ sur le plan $\mathcal{P}$.

    \item Calculer la distance de $A=(1,0,1)$ au plan $\mathcal{P}$.

    \item Donner l'expression analytique de la symétrie orthogonale $\sigma$ par rapport à $\mathcal{P}$.

  \end{enumerate}
\end{exo}

%-----------------------------------
\begin{exo} (Composition d'applications affines)

  On se place dans un plan affine réel.
  \begin{enumerate}
    \item Soient $ABC$ un triangle, $A'$, $B'$, $C'$ les milieux respectifs de $BC$, $CA$ et $AB$ , $s_{A'}$ , $s_{B'}$, $s_{C'}$ les symétries centrales par rapport à ces points. Déterminer la nature géométrique des transformations $f=s_{B'}\circ s_{A'}$ et $g=s_{C'}\circ s_{B'}\circ s_{C'}$.

    \item Soient $f$ et $f'$ deux homothéties de même rapport quelle est la nature de la transformation $f\circ f'^{-1}$ ?
  \end{enumerate}
\end{exo}

%-----------------------------------
\begin{exo} (Polygone des milieux)

  Dans cet exercice on se place dans $\mathbb{R}^{2}$:

  \begin{enumerate}
    \item Soit $A'B'C'$ un triangle. Montrer qu'il existe un triangle $ABC$, et un seul, tel que $A'$ soit le milieu de $BC$, $B'$ le milieu de $CA$ et $C'$ le milieu de $AB$. Indiquer une construction géométrique de ce triangle.

    \item Soit $A'B'C'D'$ un quadrilatère. Donner une condition nécessaire et suffisante pour qu'il existe un quadrilatère $ABCD$ tel que $A'$ soit le milieu de $AB$, $B'$ le milieu de $BC$, $C'$ le milieu de $CD$ et $D'$ le milieu de $DA$. Ce quadrilatère, s'il existe, est-il unique?

    \item\hard Etant donnés $n$ points $B_{1},\ldots,B_{n}$, peut-on toujours trouver $n$ points $A_{1},\ldots,A_{n}$ tels que $B_{i}$ soit, pour tout $i=1,\ldots,n$, le milieu de $A_{i}A_{i+1}$ (avec la convention $A_{n+1}=A_{1}$ ) ? Donner une construction géométrique des points $A_{i}$ à partir des points $B_{i}$ lorsque la solution existe.\newline
    \begin{indication}
      Comme le montrent les deux premières questions, la solution dépend de la parité de $n$. On pourra considérer la composée des symétries centrales de centres $B_{1},\ldots,B_{n}$.
    \end{indication}

  \end{enumerate}
\end{exo}

%-----------------------------------
\begin{exo} (Homothéties)

  On se place dans un plan affine réel. On rappelle que l'homothétie de centre $A$ et de rapport $\lambda$ est l'application $h$ qui à un point $M$ associe le point $h(M)$ vérifiant:
    \[
      \overrightarrow{Ah(M)}=\lambda\overrightarrow{AM}
    \]

  \begin{enumerate}
    \item Que dire des cas particuliers $\lambda=0,1,-1$? A quelle condition $h$ est-elle une bijection? Si elle est bijective quelle est son inverse?

    \item On se place dans un repère affine $\mathcal{R}=(O,I,J)$. Donner l'expression analytique de l'homothétie $h$ de centre $A=(u,v)_{\mathcal{R}}$ et de rapport $\lambda$.

    \item Calculer le conjugué d'une homothétie par une transformation affine.

    \item Calculer la composée de deux homothéties.

    \item Les homothéties de rapport non nul forment-elles un groupe?

    \item Montrer que les homothéties de même centre $A$ et de rapport non nul forment un groupe.

    \item Montrer que les homothéties et les translations forment un sous-groupe $HT_2(\mathbb{R})$ du groupe affine, puis que ce sous-groupe est engendré par les homothéties. Le groupe $HT_2(\mathbb{R})$ agit-il transitivement sur le plan affine?

    \item On rappelle que \textit{le centre} d'un groupe $G$ est l'ensemble des éléments $g\in G$ tels que $gh=hg$, $\forall h\in G$. Montrer que le centre de $G$ est un sous-groupe de $G$.

    \item Quel est le centre du groupe linéaire $GL_{2}(\mathbb{R})$?

    \item Quel est le centre du groupe affine $GA_{2}(\mathbb{R})$?

    \item A quelles conditions une homothétie et une translation commutent-elles?
  \end{enumerate}
\end{exo}


% ==================================
\section{Compléments}
% ==================================

%-----------------------------------
\begin{exo} (Quelques théorèmes classiques)

  On se place dans un plan affine réel.

  \begin{enumerate}
  \item Montrer le \textbf{théorème de Pappus} (d'après Pappus d'Alexandrie, IV\textsuperscript{e} siècle après J.-C.):

  \textit{\og Soient $\mathcal{D}$ et $\mathcal{D}'$ deux droites distinctes et des points $A,B,C\in \mathcal{D}$ et $A',B',C'\in \mathcal{D}'$
  tels que $\affspan{AB'}\parallel\affspan{A'B}$ et $\affspan{BC'}\parallel\affspan{B'C}$, alors $\affspan{AC'}\parallel\affspan{A'C}$.\fg}

  \item Montrer le \textbf{théorème de Ceva} (d'après Giovanni Ceva, 1678, même si ce théorème était connu à la fin du XI\textsuperscript{e} siècle de Yusuf Al-Mu'taman ibn Hűd, géomètre et roi de Saragosse.)

  \textit{\og Soit $ABC$ un triangle non dégénéré, soient $D$, $E$ et $F$ trois points distincts des sommets et appartenant respectivement aux segments $[BC]$, $[CA]$ et $[AB]$. Les droites $\affspan{AD}$, $\affspan{BE}$ et $\affspan{CF}$ sont concourantes si et seulement si $\frac{\overline{DB}}{\overline{DC}}\times\frac{\overline{EC}}{\overline{EA}}\times\frac{\overline{FA}}{\overline{FB}}=-1$.\fg}

  \item Montrer le \textbf{théorème de Ménélaüs} (d'après Ménélaüs d'Alexandrie, I\textsuperscript{er} et II\textsuperscript{e} siècle après J.-C.)

  \textit{\og Si $D$, $E$ et $F$ sont trois points des côtés $\affspan{BC}$, $\affspan{AC}$ et $\affspan{AB}$ d'un triangle non dégénéré $ABC$ (et distincts des sommets), alors $D$, $E$ et $F$ sont alignés si et seulement si $\frac{\overline{DB}}{\overline{DC}}\times\frac{\overline{EC}}{\overline{EA}}\times\frac{\overline{FA}}{\overline{FB}}=1$.\fg}

  \item Montrer le \textbf{théorème de Desargues} (d'après Girard Desargues, alias S.G.D.L., XVII\textsuperscript{e} siècle à Lyon):

  \textit{\og Soient $ABC$ et $A'B'C'$ deux triangles non dégénérés sans sommets communs et tels que $\affspan{AB}\parallel\affspan{A'B'}$; $\affspan{AC}\parallel\affspan{A'C'}$; $\affspan{BC}\parallel\affspan{B'C'}$, alors les droites $\affspan{AA'}$, $\affspan{BB'}$ et $\affspan{CC'}$ sont parallèles ou concourantes.\fg}

  \end{enumerate}
\end{exo}

%-----------------------------------
\begin{exo} (Ensembles convexes)

  \begin{enumerate}
    \item Soient $C$ et $C'$ deux convexes non vides d'un espace réel affine $E$. Montrer que l'ensemble $\Gamma$ des milieux des segments $MM'$ , où $M$ parcourt $C$ et $M'$ parcourt $C'$ est un convexe.

    \item Montrer que les images directes et inverses des convexes par les applications affines sont également convexes.
  \end{enumerate}
\end{exo}

%-----------------------------------
\begin{exo} (Demi-espaces)

  Soit $E$ un espace affine réel de dimension $n$ et $H$ un hyperplan affine de $E$.
  \begin{enumerate}

    \item Montrer que la relation $\sim $ définie sur $E\backslash H$ par:
      \[
        M\sim N\Longleftrightarrow[MN]\cap H=\emptyset
      \]
    est une relation d'équivalence qui sépare $E\backslash H$ en exactement deux classes, appelés \textit{demi-espaces (ouverts)}.

    \item Montrer que les demi-espaces sont convexes.
  \end{enumerate}
\end{exo}

%-----------------------------------
\begin{exo} (Caractérisation des homothéties-translations)

  Soient $\ens{E}$ un espace affine réel et $\varphi \in \aff(\ens{E})$ une application affine qui envoie toute droite de $\ens{E}$ sur une droite parallèle. Montrer que $\varphi$ est une translation ou une homothétie de rapport non nul.
\end{exo}

%-----------------------------------
\begin{exo} (Théorème fondamental de la géométrie affine)

  On fixe un espace affine réel $\mathcal{E}$ de dimension au moins $2$, on se donne une bijection $\phi$ de $\mathcal{E}$ dans $\mathcal{E}$. Le but de cet exercice est de montrer que $\phi$ est affine si et seulement si $\text{\ensuremath{\phi}}$ préserve les triplets de points alignés.

  \begin{convention}
    Dans tout ce qui suit on suppose que $\phi$ envoie $3$ points alignés sur $3$ points alignés. On note par ${}'$ les images par $\phi$: $O'=\phi(O)$, $A'=\phi(A)$, $B'=\phi(B)$, \ldots
  \end{convention}

  \begin{enumerate}
    \item Soit $\mathcal{F}$ un sous-espace affine de $\mathcal{E}$, montrer que $\phi^{-1}(\mathcal{F})$ est affine.

    \item Montrer que $\phi$ envoie un repère affine sur un repère affine.

    \item En déduire que $\phi$ envoie $k$ points affinement indépendants sur $k$ points affinement indépendants.

    \item En déduire que $\phi$ préserve droites et plans.

    \item \label{droitesplans} En déduire que $\text{\ensuremath{\phi}}$ envoie les droites parallèles sur des droites parallèles.

    \item On fixe $O$ un point de $\mathcal{E}$. Soient $A$ et $B$ deux points de $\mathcal{E}$ et soit $C$ tel que $\overrightarrow{OC}=\overrightarrow{OA}+\overrightarrow{OB}$, de plus on suppose que $A,B$ et $C$ ne sont pas alignés. En utilisant la question \ref{droitesplans}) montrer que:\vspace{-3mm}
      \[
        \overrightarrow{O'C'}=\overrightarrow{O'A'}+\overrightarrow{O'B'}
      \]

    \item Soient $\mathcal{D}$ une droite passant par $O$ et $\mathcal{D}'$ son image par $\phi$. On fixe $A\neq O$ un point de $\mathcal{D}$. Pour tout scalaire $\lambda\in\mathbb{R}$, on considère le point $M$ tel que $ \overrightarrow{OM}=\lambda\overrightarrow{OA}$.\newline
    Vérifier que son image $M'=\phi(M)$ satisfait $\overrightarrow{O'M'}=\mu\overrightarrow{O'A'}$ pour un unique scalaire $\mu$, indépendant du choix de $A$.

    On a ainsi défini une fonction
      $
        \sigma_{\mathcal{D}} :
        \begin{cases}
          \mathbb{R}\rightarrow\mathbb{R}\\
          \lambda\mapsto\mu
        \end{cases}
      $.\vspace{1mm}

    \item Soient $M$ et $N$ tels que $\overrightarrow{OM}=\lambda_{1}\overrightarrow{OA}$ et $\overrightarrow{ON}=\lambda_{2}\overrightarrow{OA}$. En utilisant un point $B$ hors de $\mathcal{D}$ et des droites parallèles, construire les points $P$ et $Q$ de $\mathcal{D}$ tels que
      \[
        \overrightarrow{OP}=(\lambda_{1}+\lambda_{2})\overrightarrow{OA},\qquad
        \overrightarrow{OQ}=(\lambda_{1}\lambda_{2})\overrightarrow{OA}.
      \]

    \item Montrer que $\sigma_{\mathcal{D}}$ vérifie $\sigma_{\mathcal{D}}(\lambda_{1}+\lambda_{2})=\sigma_{\mathcal{D}}(\lambda_{1})+\sigma_{\mathcal{D}}(\lambda_{2})$ et que $\sigma_{\mathcal{D}}(\lambda_{1}.\lambda_{2})=\sigma_{\mathcal{D}}(\lambda_{1}).\sigma_{\mathcal{D}}(\lambda_{2})$. En déduire que $\sigma_{\mathcal{D}}$ est un automorphisme du corps $\mathbb{R}$.

    \item Soit $\sigma$ un automorphisme du corps $\mathbb{R}$. Vérifier que $\sigma(1)=1$, que $\sigma(x)=x, \forall x\in\mathbb{Q}$, que $\sigma$ est une application croissante, puis en déduire que $\sigma=Id$.

    \item Montrer que $\phi$ est affine.
  \end{enumerate}
\end{exo}

\end{document}
