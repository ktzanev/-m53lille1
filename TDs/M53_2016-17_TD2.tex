\documentclass[a4paper,12pt,reqno]{amsart}
\usepackage{macros_M53}

\begin{document}

\hautdepage{TD2: Espaces euclidiens, notions de base}

% ==================================
\section{Questions métriques}
% ==================================

%-----------------------------------
\begin{exo} (Perpendiculaire commune)

  On se place dans l'espace $\mathbb{R}^{3}$.
  \begin{enumerate}
    \item Soient $\mathcal{D}_{1}$ et $\mathcal{D}_{2}$ deux droites distinctes de l'espace. Justifier l'existence d'une perpendiculaire commune à ces deux droites. Sous quelles conditions est-elle unique?

    \item Donner des équations de la perpendiculaire commune aux droites $\mathcal{D}_{1}$ d'équations\newline \{$x+y-z-1=0,\ 2x+y+z=0\}$ et $\mathcal{D}_{2}$ déterminée par le point $A$ de coordonnées $(1,0,1)$ et le vecteur directeur $\overrightarrow{u}$ de composantes $(1,-1,0)$.

    \item Quelle est la distance entre $\mathcal{D}_{1}$ et $\mathcal{D}_{2}$?
  \end{enumerate}
\end{exo}

%-----------------------------------
\begin{exo} (Distance pondérée à un ensemble de points)

  Soit $(A_i,\lambda_i)_{1\leq i\leq n}$ un système pondéré de $n$ points d'un espace affine euclidien de poids total $\sum_{i=1}^n \lambda_i$ non nul. Pour tout point $M$ on définit la fonction
    $$
      \phi(M)=\sum_{i=1}^n \lambda_i |\overrightarrow{MA_i}|^2.
    $$
  \begin{enumerate}
    \item Soit $G$ le barycentre du système pondéré $(A_i,\lambda_i)_{1\leq i\leq n}$\,, montrer que:
      $$
        \phi(M)=(\sum_{i=1}^n\lambda_i) |\overrightarrow{MG}|^2+\phi(G).
      $$
    \item Discuter des lignes de niveau de la fonction $\phi$ dans le plan et dans l'espace.
  \end{enumerate}
\end{exo}


%-----------------------------------
\begin{exo} (L'espace euclidien des matrices)

  On se place dans l'espace vectoriel $M_3(\mathbb{R})$ des matrices $3\times 3$.
  \begin{enumerate}
    \item Rappeler comment munir $M_3(\mathbb{R})$ d'une structure d'espace euclidien. En donner une base orthonormée.
    \item Donner une base orthonormée de l'espace des matrices antisymétriques.
    \item Calculer l'orthogonal des matrices antisymétriques.
    \item Calculer la distance de la matrice%
      \scalebox{.7}{
        $
          \begin{pmatrix}
            1 & 0 & 0 \\
            1 & 0 & 1 \\
            0 & 0 & 1
          \end{pmatrix}
        $
      }
    au sous-espace des matrices diagonales.
  \end{enumerate}
\end{exo}


%-----------------------------------
\begin{exo} (Théorème de Sylvester-Gallai)
  \begin{enumerate}
    \item Étant donné un nombre fini de points dans un plan affine (euclidien), montrer qu'on a l'alternative suivante:
      \begin{itemize}
        \item soit tous les points sont alignés,
        \item soit il existe une droite qui contient exactement deux points de l'ensemble.
      \end{itemize}
      \emph{Indication: Considérer un couple $(P,\mathcal{D})$, où $\mathcal{D}$ est une droite contenant au moins deux points de l'ensemble et $P\notin \mathcal{D}$ est un point de l'ensemble, tel que la distance $\mathrm{d}(P,\mathcal{D})$ est minimale.}

    \item Est-ce que la question précédente reste vraie si les points sont dans un espace de dimension quelconque?

    \item Est-ce que la première question reste vraie pour un nombre infini de points?
  \end{enumerate}
\end{exo}


% ==================================
\section{Convexes}
% ==================================

%-----------------------------------
\begin{exo} (Convexes euclidiens)

    Soient $\mathcal{C}$ un convexe fermé non vide d'un espace euclidien et $P \notin \mathcal{C}$ un point de cet espace.
    \begin{enumerate}
       \item Montrer qu'il existe un unique point $Q \in \mathcal{C}$ tel que $d(P,Q) = d(P,\mathcal{C})$. \newline
       \emph{On dit que $Q$ est la projection de $P$ sur $\mathcal{C}$.}

       \item Montrer que l'hyperplan $\mathcal{H}$ passant par $Q$ et orthogonal à $\overrightarrow{QP}$ est un \emph{plan de support}, c'est-à-dire que $\mathcal{H}$ rencontre $\mathcal{C}$, mais pas son intérieur.

       \item Montrer que tout convexe fermé peut s'écrire comme l'intersection de demi-espaces affines.

       \item Est-ce que la question précédente reste vraie si on se place dans un espace affine de dimension finie, plutôt que dans un espace euclidien?
     \end{enumerate}
\end{exo}

%-----------------------------------
\begin{exo} (Séparation de convexes)

  Étant donnés deux convexes compacts non vides disjoints dans un espace affine de dimension finie, montrer qu'il existe un hyperplan qui les sépare strictement (c'est-à-dire qu'il ne les rencontre pas et que les deux convexes ne sont pas dans le même demi-espace délimité par cet hyperplan).
\end{exo}


% ==================================
\section{Isométries}
% ==================================
%-----------------------------------
\begin{exo} (Isométries du triangle)
  \begin{enumerate}
    \item Déterminer le groupe des isométries d'un triangle dans le plan, la discussion sera menée en fonction des propriétés métriques du triangle.
    \item Même question avec un quadrilatère.
  \end{enumerate}
\end{exo}

%-----------------------------------
\begin{exo} (Isométries du tétraèdre)

    Montrer que le groupe des isométries affines qui préserve un tétraèdre régulier est isomorphe à $\mathfrak{G}_{4}$, le groupe de permutations de $4$ points.
\end{exo}

%-----------------------------------
\begin{exo} (Nature de certaines isométrie)
  \begin{enumerate}
    \item Soit l'isométrie linéaire%
      \scalebox{.7}{
        $
          \begin{pmatrix}
            0 & 0 & 1 \\
            1 & 0 & 0 \\
            0 & 1 & 0
          \end{pmatrix}
        $.
      }
      Décrire la nature de cette isométrie.
    \item Soient $R$ et $T$ une rotation et une translation de $\mathbb{R}^{3}$. Quelle est la nature de $T \circ R$ ?
    \item Quelle est la composée de trois symétries de plans parallèles de $\mathbb{R}^{3}$?
  \end{enumerate}
\end{exo}

%-----------------------------------
\begin{exo} (Capes 2007, 2\ieme\ épreuve)
% Pour voir l'image originale
% \usepackage{graphicx}
% \hspace{-7mm}\includegraphics[scale=0.55]{CAPES2007.png}

\begin{center}
  \textbf{Partie V. GROUPES DIÉDRAUX}
\end{center}

\begin{enumerate}
  \renewcommand{\theenumi}{\arabic{enumi}}
  \renewcommand{\labelenumi}{\bf\theenumi.}
  \renewcommand{\theenumii}{\alph{enumii}}
  \renewcommand{\labelenumii}{\theenumii)}

  \item Soit $E$ un plan affine euclidien orienté. Soit $p \in \mathbb{N}$, $p \geq 2$. On appelle groupe diédral d'ordre $2p$, noté $D_{2p}$, le groupe des isométries laissant invariant un polygone régulier
  \[
    \mathcal{P}_{p}=\left\{M_{0},\ldots,M_{p-1}\right\}
  \]
  à $p$ sommets, parcourus dans le sens direct. On pose $M_{p}=M_{0}$.
  \begin{enumerate}
    \item Montrer que le sous-groupe $C_{p}$ de $D_{p}$ constitué des isométries directes est un groupe cyclique d'ordre $p$ engendré par la rotation $\rho$ de centre $O$ et d'angle $\frac{2 \pi}{p}$, où $O$ est le centre du polygone $\mathcal{P}_{p}$.

    \item Préciser une symétrie orthogonale $\sigma$ laissant le polygone $\mathcal{P}_{p}$ invariant.

    \item Montrer que
    \[
      D_{2p} = \left\{
        \rho^{i}\circ \sigma^{j} \,;\, i \in \{1,\ldots,p-1\} \text{ et } j \in \{0,1\}
      \right\}
    \]
    et en déduire que $D_{2p}$ est un groupe d'ordre $2p$.

    \item Soit $k \in \{1,\ldots,p-1\}$. Montrer que $\sigma \circ \rho^{k} \circ \sigma = \rho^{p-k}$.
  \end{enumerate}
\end{enumerate}

\end{exo}

% ==================================
\section{Géométrie projective sans le dire}
% ==================================

%-----------------------------------
\begin{exo} (points cocycliques et le théorème de Ptolémée)

  On se donne $4$ points distincts $A$, $B$, $C$ et $D$ du plan complexe d'affixes $a$, $b$, $c$ et $d$ respectivement.
  \begin{enumerate}
    \item Montrer que l'on a toujours
      $$
        |AC|.|BD| \leq |AB|.|CD|+|AD|.|BC|
      $$
      avec égalité si et seulement s'il existe un réel $\lambda > 0$ tel que $(b-a)(d-c)=\lambda (d-a)(c-b)$.
    \item Déduire de la question précédente que le cas d'égalité est équivalent à
      $$
        \arg(\frac{b-a}{d-a})-\arg(\frac{b-c}{d-c}) = \pm\pi.
      $$
    \item Montrer le théorème de Ptolémée :

    \textit{Un quadrilatère convexe est inscrit dans un cercle si et seulement si le produit des longueurs de ses diagonales est égal à la somme des produits des longueurs des côtés opposés.}
  \end{enumerate}
 \end{exo}


%-----------------------------------
\begin{exo} (Birapports)

  Soient $4$ points du plan complexe avec affixes $a$, $b$, $c$, $d$ distincts deux à deux. On définit leur birapport comme étant la quantité:
  $$
    [a,b,c,d]=\frac{a-c}{b-c}\cdot\frac{b-d}{a-d} \in \mathbb{C}.
  $$
  \begin{enumerate}
    \item Montrer que si $4$ points sont alignés alors leur birapport est un nombre réel. Dans quel cas ce birapport est positif ?
    \item Montrer que si $4$ points sont cocycliques alors leur birapport est un nombre réel. Dans quel cas ce birapport est positif ?
    \item Montrer que le birapport de $4$ points est réel si et seulement si les $4$ points sont alignés ou cocycliques.
    \item Prouver que si $4$ points d'affixes non nulles sont alignés alors leurs images par $1/\overline{z}$ sont alignées ou cocycliques.
    \item Montrer que les homothéties de rapport non nul, les rotations et les translations préservent les birapports, et que les symétries les transforment en leurs conjugués.
    \item Montrer que les points d'affixes $-2+6i$, $1+7i$, $4+6i$, $6+2i$ sont cocycliques.
  \end{enumerate}
\end{exo}


%-----------------------------------
\begin{exo} (Inversions)

  Une inversion $i(\Omega,r)$ de centre $\Omega$ et de rapport $r \in \mathbb{R}^{*}_{+}$ est une application du plan $P$ privé de $\Omega$ dans $P$ qui à tout point $M$ associe le point $M'$ tel que
  \begin{itemize}
    \item $M'$ soit sur la droite $\Omega M$,
    \item $\overline{\Omega M}.\overline{\Omega M'}=r^2$.
  \end{itemize}
  \begin{enumerate}
    \item Déterminer l'image de $i(\Omega,r)$.
    \item Montrer que l'inversion est une involution.
    \item En identifiant le plan $P$ avec $\mathbb{C}$, donner l'expression analytique de l'inversion $i(0,1)$, puis, plus généralement, de $i(\omega,r)$ avec $\omega \in \mathbb{C}$ et $r \in \mathbb{R}^{*}_{+}$.
    \item  Montrer que chaque point du cercle de centre $\Omega$ et de rayon $r$ est invariant par l'inversion.
    \item Montrer qu'une inversion transforme un birapport en son conjugué.
    \item Montrer que l'inversion préserve les droites qui passent par $\Omega$.
    \item Montrer qu'elle envoie une droite qui ne passe par $\Omega$ sur un cercle qui passe par $\Omega$.
    \item Montrer aussi qu'elle envoie tout cercle qui ne passe pas par $\Omega$ sur un cercle qui ne passe pas par $\Omega$.
  \end{enumerate}
\end{exo}


%-----------------------------------
\begin{exo} (Homographies)

  Une homographie est une application de $\mathbb{C}$ dans $\mathbb{C}$ de la forme
  $$
    z\mapsto \frac{az+b}{cz+d}
  $$
  avec $a,b,c,d$ complexes, tels que $ad-bc\neq 0$.
  \begin{enumerate}
    \item Montrer que toute homographie est la composée de translations, d'homothéties et au plus d'une inversion et d'une réflexion.
    \item Montrer que toute homographie envoie droites ou cercles sur droites ou cercles.
    \item Montrer que les homographies forment un groupe.
  \end{enumerate}
\end{exo}


%-----------------------------------
\begin{exo}

  \image{r}{7cm}{-17mm}{0mm}{M53_2016-17_TD2_exo15.tikz}

  Deux cercles $\mathcal{C}$ et $\mathcal{C}'$ sont tangents intérieurement en $O$. On considère une famille de cercles deux à deux tangents entre eux et tangents à $\mathcal{C}$ et $\mathcal{C}'$. Montrer que les points de contact de ces cercles entre eux sont situés sur un même cercle tangent en $O$ aux deux cercles $\mathcal{C}$ et $\mathcal{C}'$.
\end{exo}


\end{document}
