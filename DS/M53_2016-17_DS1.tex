\documentclass[a4paper,12pt,reqno]{amsart}
\usepackage{graphicx}
\usepackage{macros_M53}

% pour voir les solutions il faut enlever le commentaire de la ligne suivante
% \solutionstrue

\begin{document}

% ==================================
\hautdepage{

\ifsolutions{Solutions de l'interrogation}\else{Interrogation}\fi\par\normalfont\normalsize
25 octobre 2016\\{[ durée: 2 heures ]}\par
}
% ==================================
\ifsolutions\else
% {\fontencoding{U}\fontfamily{futs}\selectfont\char 66\relax}
\tikz[baseline=(e.base)]{\NoAutoSpacing\node(e){!};\draw[red,ultra thick,line join=round,yshift=-.15ex](90:1em)--(210:1em)--(330:1em)--cycle;}
\textbf{Documents autorisés :}\textit{Une feuille A4 recto-verso écrite à la main.}

\vspace{14mm}
\fi



%-----------------------------------
\begin{exo} (Transformations affines)

  On note $h_{\Omega,\lambda}$ l'homothétie de centre $\Omega$ et de rapport $\lambda$ dans l'espace affine $\ens{E}$. Soit $\Omega_{1}$ et $\Omega_{2}$ deux points de $\ens{E}$ et $\lambda_{1}$ et $\lambda_{2}$ deux nombres réels tels que $\lambda_{1}\lambda_{2}=1$.

  \begin{questions}
    \item Montrer que la composée $T=h_{\Omega_{1},\lambda_{1}}\circ h_{\Omega_{2},\lambda_{2}}$ est une translation.
    \item Exprimer le vecteur de translation $\vv{v}$ de $T$ en fonction des $\Omega_{1}$,$\Omega_{2}$,$\lambda_{1}$ et $\lambda_{2}$.
    \item  Illustrer la composée $T$ sur un dessin.
  \end{questions}

\end{exo}

\begin{solution}
  \begin{enumerate}
    \item Nous avons la partie linéaire $\vv{T}=\vv{h_{\Omega_{1},\lambda_{1}}}\circ \vv{h_{\Omega_{2},\lambda_{2}}}=\lambda_{1}\id \circ \lambda_{2}\id = \lambda_{1}\lambda_{2}\id = \id$, donc d'après le cours $T$ est une translation.
    \item Soit $M \in \ens{E}$, en utilisant que $h_{\Omega,\lambda}(M) = (1-\lambda)\Omega + \lambda M$, nous avons $T(M) = h_{\Omega_{1},\lambda_{1}}\big((1-\lambda_{2})\Omega_{2} + \lambda_{2}M\big)=(1-\lambda_{1})\Omega_{1} + \lambda_{1}\big((1-\lambda_{2})\Omega_{2} + \lambda_{2}M\big)= (1-\lambda_{1})\Omega_{1} + (\lambda_{1}-\lambda_{1}\lambda_{2})\Omega_{2} + \lambda_{1}\lambda_{2}M = M + (\lambda_{1}-1)\vv{\Omega_{1}\Omega_{2}}$. Ainsi on trouve que $\vv{v} = (\lambda_{1}-1)\vv{\Omega_{1}\Omega_{2}}$.
    % Soit $\Omega_{2}=\Omega_{1}+\vv{w}$ avec $\vv{w}=\vv{\Omega_{1}\Omega_{2}}$. Comme $h_{\Omega_{2},\lambda_{2}}(\Omega_{2})=\Omega_{2}$, nous avons $T(\Omega_{2}) = h_{\Omega_{1},\lambda_{1}}(\Omega_{2}) = h_{\Omega_{1},\lambda_{1}}(\Omega_{1}+\vv{w})=\Omega_{1}+\lambda_{1}\vv{w} = \Omega_{2}-\vv{w}+\lambda_{1}\vv{w}$. Ainsi on trouve que $\vv{v} = (\lambda_{1}-1)\vv{w} = (\lambda_{1}-1)\vv{\Omega_{1}\Omega_{2}}$.
    \item Sur le figure ci-dessous nous avons illustré le cas $\lambda_{1} \in\, ]0,1[\;  \Leftrightarrow\lambda_{2} > 1$.
    \begin{center}
      \begin{tikzpicture}
  \def\l{.3}
  \path
    (0,0) coordinate (O2) node[scale=2]{.} node[below]{$\Omega_2$}
    (3,0) coordinate (O1) node[scale=2]{.} node[below]{$\Omega_1$}
    (50:4) coordinate (M1) node[scale=2]{.} node[above]{$h_{\Omega_2,\lambda_2}(M)$}
    ($(O2)!\l!(M1)$) coordinate (M) node[scale=2]{.} node[left]{$M$}
    ($(O1)!\l!(M1)$) coordinate (M2) node[scale=2]{.} node[right]{$h_{\Omega_1,\lambda_1}\circ h_{\Omega_2,\lambda_2}(M)$}
  ;
  \path[-latex] (O2) edge (O1) (M) edge node[above]{$\vv{v}$} (M2);
  \path[bend left,dashed,-latex,gray] (M) edge (M1) (M1) edge (M2);
  \draw (O2) -- (M1) -- (O1);
\end{tikzpicture}

    \end{center}
  \end{enumerate}
\end{solution}
%-----------------------------------
\begin{exo} (Sous-espaces affines)

  \begin{enumerate}
    \item (Question de cours) Démontrer le résultat suivant vu en cours :

    \vspace{7pt}
    \framebox{
      \parbox{\textwidth-35mm}{\vspace{1mm}
        \emph{%
          Soient $\ens{E}$ et $\ens{F}$ deux espaces affines, $\ens{H}$ un sous-espace affine de $\ens{E}$ et $\phi \in \aff(\ens{E},\ens{F})$ une application affine.\\
          Alors l'image $\phi(\ens{H})$ de $\ens{H}$ par $\phi$ est un sous-espace affine de $\ens{F}$ de direction $\vv\phi(\ev{H})$, l'image de la direction $\ev{H}$ de $\ens{H}$ par la partie linéaire $\vv\phi$ de $\phi$.
        }
      }
    }
    \vspace{7pt}

    \item Montrer que l'ensemble $\mathbb{U}_{3}$ des polynômes unitaires de degré 3 (c.-à-d. dont le terme de plus haut degré est $X^{3}$) est un sous-espace affine de l'espace vectoriel $\mathbb{R}_{3}[X]$ des polynômes de degré au plus 3.

    \item Donner un repère cartésien, puis un repère affine de $\mathbb{U}_{3}$.

    \item On considère l'application $\delta:\mathbb{R}_{3}[X] \rightarrow \mathbb{R}_{2}[X]$, $P \mapsto P'$ qui associe à un polynôme $P \in \mathbb{R}_{3}[X]$ sa dérivé $\delta(P)=P' \in \mathbb{R}_{2}[X]$.\\
    Montrer que l'image $\delta(\mathbb{U}_{3})$ de $\mathbb{U}_{3}$ par $\delta$ est un sous-espace affine et déterminer sa direction.

    \item Donner un repère cartésien, puis un repère affine de $\delta(\mathbb{U}_{3})$.
   \end{enumerate}

\end{exo}

\begin{solution}
  \begin{enumerate}
    \item On fixe $\Omega \in \ens{H}$, ainsi $\ens{H} = \Omega + \ev{H}$. Nous avons $\phi(\ens{H}) = \{ \phi(\Omega + \vv{v}) \mid \vv{v} \in \ev{H}\} = \{ \phi(\Omega) + \vv\phi(\vv{v})) \mid \vv{v} \in \ev{H}\} = \phi(\Omega) + \vv\phi(\ev{H})$. Et comme $\vv\phi(\ev{H})$ est un sous-espace vectoriel de $\ev{F}$, comme l'image par une application linaire du sous-espace vectoriel $\ev{H}$ de $\ev{E}$, on peut conclure que $\phi(\ens{H})$ est un sous-espace affine de direction $\vv\phi(\ev{H})$.
    \item Par définition tout polynôme $P$ de $\mathbb{U}_{3}$ est de la forme $P(X)=X^{3}+R(X)$ avec $R \in \mathbb{R}_{2}[X]$. Ainsi $\mathbb{U}_{3} = X^{3} + \mathbb{R}_{2}[X]$ et comme $\mathbb{R}_{2}[X]$ est un sous-espace vectoriel de $\mathbb{R}_{3}[X]$, nous pouvons conclure que $\mathbb{U}_{3}$ est un sous-espace affine de $\mathbb{R}_{3}[X]$ de direction $\vv{\mathbb{U}_{3}}=\mathbb{R}_{2}[X]$.
    \item Comme $\{1,X,X^2\}$ est une base de $\mathbb{R}_{2}[X]$, nous avons le repère cartésien $\{\Omega,\vv{e_{0}},\vv{e_{1}},\vv{e_{2}}\}$ avec $\Omega = X^{3} \in \mathbb{U}_{3}$ et $\vv{e_{i}}=X^{i} \in \mathbb{R}_{2}[X]$ pour $i=0,1,2$. Ainsi $\{\Omega,\Omega+\vv{e_{0}},\Omega+\vv{e_{1}},\Omega+\vv{e_{2}}\} = \{X^{3},X^{3}+1,X^{3}+X,X^{3}+X^{2}\}$ est un repère affine.
    \item Comme la dérivation $P \mapsto P'$ est une application linéaire, alors $\delta$ est une application affine dont la partie linéaire est $\delta$ lui-même. Ainsi d'après la première question $\delta(\mathbb{U}_{3})$ est un sous-espace affine de direction $\delta(\vv{\mathbb{U}_{3}})=\delta(\mathbb{R}_{2}[X]) = \mathbb{R}_{1}[X]$ (les dérivé des polynômes de degré $\leq 2$ sont les polynômes de degré $\leq 1$).
    \item Soit $\Omega'=\phi(\Omega)=3X^{2} \in \delta(\mathbb{U}_{3})$. Comme $\{\vv{e_{0}},\vv{e_{1}}\}=\{1,X\}$ est une base de $\mathbb{R}_{1}[X]$ nous avons le repère cartésien $\{\Omega',\vv{e_{0}},\vv{e_{1}}\} = \{3X^{2},1,X\}$ et le repère affine $\{\Omega',\Omega'+\vv{e_{0}},\Omega'+\vv{e_{1}}\} = \{3X^{2},3X^{2}+1,3X^{2}+X\}$.
  \end{enumerate}

\end{solution}

\tsvp
\ifsolutions\else\newpage\fi

%-----------------------------------
\begin{exo} (Géométrie dans la plan complexe)

  On se place dans le plan complexe $\mathbb{C}$. On considère deux points $A$ et $B$ d'affixes respectives $2i$ et $1-i$.

  \begin{enumerate}
    \item Donner l'équation de la droite qui passe par $A$ et $B$ sous la forme
    $$
      \overline{\beta}z+\beta\overline{z}+\gamma=0,
    $$
    où $\beta$ et $\gamma$ sont des constantes à déterminer.
    \item Donner, sous la même forme que dans la question précédente, l'équation de la droite orthogonale à $AB$ qui passe par le milieu du segment $AB$.
    \item Donner l'équation du cercle de diamètre $AB$ sous la forme
      $$
        z\overline{z}-a\overline{z}-\overline{a}z+c=0,
      $$
      où $a$ et $c$ sont des constantes à déterminer.
  \end{enumerate}

  \vspace{4mm}
  Pour la suite de l'exercice on considère un point $C$ du cercle de diamètre $AB$. On note $z$ l'affixe de $C$.\\
  \image{r}{5cm}{7mm}{0mm}{M53_2016-17_DS1_exo2.tikz}
  \vspace{-11mm}

  \begin{enumerate}[resume]
    \item Écrire l'affixe de $C$ sous la forme $z = o + \rho e^{i\theta}$ où $o$ et $\rho$ sont deux constantes à déterminer et $\theta$ est un paramètre réel.
    \item Soit $M$ le milieu du segment $AC$. Déterminer l'affixe de $M$ en fonction de $\theta$. Puis, montrer que $M$ décrit un cercle $\ens{S}$, dont on précisera le centre et le rayon, quand $C$ décrit le cercle de rayon $AB$.
    \item On considère le rectangle positivement orienté $ACPQ$ (voir le dessin). Soit $R$ son centre. Déterminer l'affixe de $R$ en fonction de $\theta$. Puis, montrer que $R$ décrit un cercle, dont on précisera le centre $T$ et le rayon, quand $C$ décrit le cercle de rayon $AB$.
    \item Montrer que le centre $T$ du cercle décrit par $R$ se trouve sur le cercle $\ens{S}$ décrit par $M$.
  \end{enumerate}

\end{exo}

\begin{solution}
  \begin{enumerate}
    \item Dans l'équation $\overline{\beta}z+\beta\overline{z}+\gamma=0$, $\beta$ est l'affixe d'un vecteur normal à la droite. Ainsi comme l'affixe de $\vv{AB}$ est $1-3i$, on peut prendre $\beta = i(1-3i) = 3+i$. Pour déterminer $\gamma$ il suffit d'utiliser que $A$ est un point de la droite et donc $\overline{(3+i)}2i+(3+i)\overline{2i}+\gamma=0$ $\Longrightarrow$ $\gamma = -4$. Ainsi nous avons trouvé l'équation
    $$
      (3-i)z + (3+i)\overline{z} - 4 = 0.
    $$
    \item Comme $\vv{AB}$ est un vecteur normal à la droite recherchée, son équation est de la forme $\overline{(1-3i)}z+(1-3i)\overline{z}+\gamma'=0$. Et comme le milieu de $AB$ d'affixe $\frac{1-i+2i}{2}=\frac{1+i}{2}$ est sur la droite, on trouve $(1+3i)(\frac{1+i}{2})+(1-3i)(\frac{1-i}{2})+\gamma'=0$ $\Longrightarrow$ $\gamma' = 2$. Ainsi nous avons trouvé l'équation
    $$
      (1+3i)z + (1-3i)\overline{z} + 2 = 0.
    $$
    \item Dans l'équation recherchée $a$ est l'affixe du centre, et donc dans notre cas c'est le milieu de $AB$ qui a pour affixe $\frac{1+i}{2}$. Pour déterminer $c$ il suffit d'utiliser que $A$ est sur le cercle: $2i\overline{2i}-(\frac{1+i}{2})\overline{2i}-\overline{(\frac{1+i}{2})}2i+c=0$ $\Leftrightarrow$ $4+(1+i)i-(1-i)i+c=0$ $\Leftrightarrow$ $c=-2$. Ainsi nous avons trouvé l'équation
    $$
      z\overline{z}-(\tfrac{1+i}{2})\overline{z}-(\tfrac{1-i}{2})z-2=0.
    $$
    \item Soit $\Omega=\frac{1}{2}A+\frac{1}{2}B$ le milieu de $AB$ d'affixe $\frac{1+i}{2}$. Comme $C=\Omega +\vv{\Omega C}$ avec $\norm{\vv{\Omega C}}=\frac{1}{2}\norm{\vv{AB}}=\frac{1}{2}|1-3i|=\sqrt{\frac{5}{2}}$, nous avons $z=\frac{1+i}{2}+\sqrt{\frac{5}{2}}e^{i\theta}$ où $\theta$ est l'argument de $\vv{\Omega C}$.
    \item Comme $M=\frac{1}{2}A+\frac{1}{2}C$, son affixe est $\frac{1}{2}2i+\frac{1}{2}(\frac{1+i}{2}+\sqrt{\frac{5}{2}}e^{i\theta})=\frac{1+5i}{4}+\sqrt{\frac{5}{8}}e^{i\theta}$. Donc quand $C$ parcourt le cercle de diamètre $AB$, $\theta$ parcourt $[0,2\pi]$ et $M$ parcourt le cercle de centre $\frac{1+5i}{4}$ et de rayon $\sqrt{\frac{5}{8}}$.
    \item La multiplication par $\frac{1+i}{2}=\frac{1}{\sqrt{2}}e^{i\frac{\pi}{4}}$ correspond à la composé d'une rotation à $\frac{\pi}{4}$ et d'une homothétie vectorielle de rapport $\frac{1}{\sqrt{2}}$. Et comme l'affixe de $\vv{AC}$ est $\frac{1+i}{2}+\sqrt{\frac{5}{2}}e^{i\theta}-2i=\frac{1-3i}{2}+\sqrt{\frac{5}{2}}e^{i\theta}$ l'affixe de $\vv{AR}$ est $(\frac{1+i}{2})\big(\frac{1-3i}{2}+\sqrt{\frac{5}{2}}e^{i\theta}\big)=\frac{2-i}{2}+\frac{\sqrt{5}}{2}e^{i(\theta+\frac{\pi}{4})}$. Et finalement l'affixe de $R$ est $2i+\frac{2-i}{2}+\frac{\sqrt{5}}{2}e^{i(\theta+\frac{\pi}{4})}=\frac{2+3i}{2}+\frac{\sqrt{5}}{2}e^{i(\theta+\frac{\pi}{4})}$. Donc quand $C$ parcourt le cercle de diamètre $AB$, $R$ parcourt le cercle de centre $\frac{2+3i}{2}$ et de rayon $\frac{\sqrt{5}}{2}$.
    \item On doit vérifier que le point d'affixe $\frac{2+3i}{2}$ est sur le cercle de centre $\frac{1+5i}{4}$ et de rayon $\sqrt{\frac{5}{8}}$. En effet nous avons $|\frac{2+3i}{2}-\frac{1+5i}{4}|=|\frac{3+i}{4}|=\sqrt{\frac{5}{8}}$.
  \end{enumerate}

\end{solution}

\end{document}
